%----------------------------------------------------------------------------------------
%	PACKAGES AND OTHER DOCUMENT CONFIGURATIONS
%----------------------------------------------------------------------------------------

\documentclass[11pt]{letter} % Default font size of the document, change to 10pt to fit more text

%\usepackage{newcent} % Default font is the New Century Schoolbook PostScript font 
%\usepackage{helvet} % Uncomment this (while commenting the above line) to use the Helvetica font

% packages
\usepackage{hyperref} % for links
\usepackage[bottom]{footmisc} % footnotes at the bottom.
\usepackage{color} % for colors

% Margins
\topmargin=-1in % Moves the top of the document 1 inch above the default
\textheight=9in % Total height of the text on the page before text goes on to the next page, this can be increased in a longer letter
\oddsidemargin=-10pt % Position of the left margin, can be negative or positive if you want more or less room
\textwidth=6.5in % Total width of the text, increase this if the left margin was decreased and vice-versa

%\let\raggedleft\raggedright % Pushes the date (at the top) to the left, comment this line to have the date on the right
\date{}


\usepackage{color} % for colors
\usepackage{hyperref} % for links

\hypersetup{
  colorlinks = true,
  urlcolor = blue,
  pdfpagemode = UseNone
}



\begin{document}

%----------------------------------------------------------------------------------------
%	ADDRESSEE SECTION
%----------------------------------------------------------------------------------------

\begin{letter}{} 

%----------------------------------------------------------------------------------------
%	YOUR NAME & ADDRESS SECTION
%----------------------------------------------------------------------------------------

\begin{center}
\large\bf Hector Bahamonde \\ % Your name
Rutgers University\\
\vspace{20pt} \hrule height 1pt % If you would like a horizontal line separating the name from the address, uncomment the line to the left of this text
89 George St. \\ New Brunswick, NJ 08901 \\ (732) 318-9650 \\ 
{\normalfont\normalsize\href{mailto:hector.bahamonde@rutgers.edu}{Hector.Bahamonde@Rutgers.edu}} \\
{\normalfont\normalsize\href{http://www.hectorbahamonde.com}{www.HectorBahamonde.com}}\\
{\normalfont \scriptsize{
% Date
\vspace{5mm}\today\\
Download last version \href{http://github.com/hbahamonde/Job_Market/raw/master/Bahamonde_Research_Statement.pdf}{\texttt{here}}}} % Link to last version
\\
{\huge\vspace{8mm} Research Statement}
\end{center} 

%\signature{\vspace{6cm}h.b., fall 2016} % Your name for the signature at the bottom

%----------------------------------------------------------------------------------------
%	LETTER CONTENT SECTION
%----------------------------------------------------------------------------------------
\opening{} 
 
Within comparative politics, my research explores the intersection between inequality and political development, from a political economy perspective. I use cutting-edge quantitative methods to analyze the role of inequality on state capacities, comparative historical analysis to trace the origins of institutional investments, and experimental methods to study individual preferences on vote-buying. I conduct research primarily in the context of Latin America, but also have projects on vote-selling and voting behavior in the U.S. 


% job market paper // motivation.
My {\bf job market paper} is entitled \href{https://github.com/hbahamonde/IncomeTaxAdoption/raw/master/Bahamonde_IncomeTaxAdoption.pdf}{``Structural Transformations and The Political Roots of Fiscal Capacities in Latin America, 1900-2010''}. It argues that the modernization of the fiscal apparatus was the product of an inter-elite sectoral conflict process between the agricultural and industrial classes. The motivation of this project is that according  to many political economists, fiscal sociologists, development economist and economic historians, fiscal capacities are a prerequisite for ``strong'' states. Yet, the \emph{origins} of fiscal capacities are still unclear. 

% job market paper // methodology \\ research design // specific findings
I leverage sectoral outputs dating back to 1900 until 2010 of a sample of seven Latin American countries. Using panel data methods, I model the conditional hazard ratio that a country which has not yet adopted the income tax adopts it in a given year as a function of the relative size of the agricultural and industrial sectors. Within the multiple failure-time framework, I use a slightly different data structure and estimate the \emph{joint} occurrence of income taxation \emph{and} democratization. I find that as the size of the industrial sector grows, the likelihood of income taxation \emph{and} democratization grows exponentially. Using Chile as a case study, I observe that industrial growth altered the political and military balance, endangering the political monopoly the traditional sector had inherited after the colonial period. 

% job market paper // argument.
This paper is embedded into a larger book manuscript where I analyze how these structural transformations helped states to make institutional investments that lead to the formation of states with higher capacities. I utilize fine-grained historical case study comparisons, sectoral outputs from 1900 to the present, time-series econometric techniques, hazard models, and a {\bf novel earthquake dataset} that covers sub-national death tolls from 1900 to the present to measure state capacities. The manuscript builds on the fiscal sociology literature and the dual-sector economy model. My findings strongly suggest that national industrialization processes challenged the traditional sector, creating the incentives to self-impose state institutions such as skilled bureaucracies and semi-competitive congresses (oligarchic republics). This structural transformation and the subsequent institutional investments gave way to the modernization of the state and their respective societies in a way that differs entirely from modernization theory.


{\bf Book manuscript}, \emph{``Structural Transformations in Latin America: State Building and Elite Competition 1850-2010''}. The book manuscript is divided into four parts. The first part develops a theory for the origins of fiscal capacities, which is the topic of the job market paper. The second part argues that though sectoral conflicts produced political outcomes, such as investments in institutions of state building, the \emph{origins} of these conflicts were deeply rooted in the economic structure. Low-productive agricultural economies during the 1900s, by definition, could not retain labor, freeing cheap labor which the modern sector could rely on. On the contrary, by the law of comparative advantages, productive economies were able to retain labor, restricting the development of the modern sector. Structurally, this means that industrialization consisted of a shift of resources from agriculture to manufacturing. In practice, it meant that productive agricultural sectors not only reinforced the agricultural sector's political monopoly, but also that the labor transference associated to the industrialization process endangered the political leverage the agricultural sector inherited from the colonial period. I test this theory using time series econometric methods (vector error correction equations) and Granger causality tests to argue that the channel by which the industrial sector was expanded had to do with an initial consolidation of the agriculture sector followed by an even faster decay of it.


% third part: state capacities.
The third part links these two structural transformations with actual state capacities. One of the biggest gaps in the literature is that there is not a consensus regarding how to measure state capacities. Another novel implication of my book manuscript is that it proposes measuring this concept using original earthquake data. The rationale behind this measurement is very intuitive: good-quality building codes and zoning laws are a \emph{reflection} of other \emph{state institutions}. For example, \emph{Why does a 7.0 earthquake flatten Haiti leaving at least 100,000 deaths while a 8.8 earthquake in Chile in the same year leaves just 525 deaths?} The answer is that in Chile, there are strong and heavily enforced regulations. Every building before being erected has to go through a very exhaustive approval process. \emph{Haiti lacks state capacities to implement and enforce such regulations}. My project has received two grants that have allowed me to gather the data. By exploiting this historical data from 1900 to 2010 on earthquake death tolls, magnitudes and local population (to properly weight the number of deaths) for all southern cone Pacific coast Latin American countries, I provide evidence within states over time by estimating state-of-the-art spatial regression analyses to understand to which degree inter-elite conflicts (measured with sectoral output data) cause state capacities (proxied by death tolls associated to similar earthquakes). 

% fourth part: case studies and small-N section.
The fourth section of my book explores the micro-mechanisms of inter-elite disputes and state capacities within a small-N design. Here, I compare similar subnational contiguous units, for example towns in northern Chile and southern Peru, and observe in more detail the mechanics between inter-elite disputes and state-capacities when both units are affected by similar exogenous shocks (i.e. earthquakes). Keeping everything else constant, the difference in death tolls should be attributed to the degree in which the agriculture sector was challenged by the modern sector. To test this theory, I rely on process tracing, comparative historical analysis and archival analyses. 

% Vote-Buying in Brazil// Frame myself as a scholar of inequality 
{\bf Projects and Working Papers.} In addition to the book manuscript, I am currently expanding the findings of a series of papers related to vote-buying and vote-selling, using both observational and experimental data in the Americas. 

% Vote-Buying in Brazil// explaining the paper
I have a working \href{https://github.com/hbahamonde/Clientelism_paper/raw/master/Bahamonde_Clientelism_Paper.pdf}{paper} on vote-buying in Brazil ready to submit for publication. The paper starts by recognizing that there is not an agreement on whether parties target groups or individuals. In fact, most of the times, scholars assume that group-targeting and individual-targeting are interchangeable. The paper develops and tests a theory where parties make use of simultaneous segmented targeting techniques. Groups are preferable by brokers when party machines need to secure higher levels of electoral support, relying on economies of scale and spillover effects that these groups provide. However, individuals are better targets when they are more identifiable, that is when poor individuals are nested in non-poor contexts or when non-poor individuals are nested in poor groups. The theory also explains why non-poor individuals are also targeted. The paper uses observational data, a matched design and a short case study (Brazil) to confirm the assumptions made in the models.


{\bf Vote-Selling.} With the support of a generous grant, I utilized an experimental design to carry out two experiments in the U.S. out of a series of experiments to be fielded in Latin America for further comparison. I look at price elasticities on vote-selling relative to individual democratic values. Employing a \emph{list experiment} to capture non-biased answers on socially-condemnable/illegal behaviors (like vote-buying), my identification strategy allowed me to observe the ``tipping point'' where individuals prefer the cash over being a democratic citizen. In a separate study, I designed a \emph{conjoint experiment} to identify which of Dahl's democratic dimensions should fail to predict individual propensities of vote-selling. Conjoint experiments allow researchers to directly isolate complex multi-dimensional concepts (such as \emph{support for democracy}) and observe which dimension(s) is/are associated to the outcome of interest (vote-selling). Preliminary results show that when the \emph{liberal} component, particularly the \emph{right to associate} fails, individuals are more likely to sell. In this sense, this project is very innovative as it departs from the common strategy of correlating the standard support for democracy question with vote-buying. My identification strategy is more comprehensive as it \emph{decomposes} ``support for democracy'' in several dimensions which are theoretically and substantively relevant (i.e., Dahl's democracy dimensions). 


{\bf Summary and Future Research} In sum, my book manuscript, job market paper and working papers on vote-buying and experiments on vote-selling, work toward exploring the effects of inequality on political development from a comparative perspective. I use a widely broad methodological perspective, historical comparisons, time series analyses, and experimental and quasi-experimental methodologies. My goal is to use this toolkit to keep asking ``big'' questions that are fundamental for our discipline. My future research will seek to study the connection between elite competition and democratic regimes, exploring the connection between state building and democratic institutions and its relationship with vertical accountability and the role of the middle class on bureaucracy formation.

\closing{{\color{white}empty here}}


\encl{Curriculum vitae, teaching statement, cover letter, writing samples.}

%----------------------------------------------------------------------------------------

\end{letter}

\end{document}
