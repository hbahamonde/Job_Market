%----------------------------------------------------------------------------------------
%	PACKAGES AND OTHER DOCUMENT CONFIGURATIONS
%----------------------------------------------------------------------------------------

\documentclass[11pt]{letter} % Default font size of the document, change to 10pt to fit more text

%\usepackage{newcent} % Default font is the New Century Schoolbook PostScript font 
%\usepackage{helvet} % Uncomment this (while commenting the above line) to use the Helvetica font

% packages
\usepackage{hyperref} % for links
\usepackage[bottom]{footmisc} % footnotes at the bottom.
\usepackage{color} % for colors

% Margins
\topmargin=-1in % Moves the top of the document 1 inch above the default
\textheight=9in % Total height of the text on the page before text goes on to the next page, this can be increased in a longer letter
\oddsidemargin=-10pt % Position of the left margin, can be negative or positive if you want more or less room
\textwidth=6.5in % Total width of the text, increase this if the left margin was decreased and vice-versa

%\let\raggedleft\raggedright % Pushes the date (at the top) to the left, comment this line to have the date on the right
\date{}




\begin{document}

%----------------------------------------------------------------------------------------
%	ADDRESSEE SECTION
%----------------------------------------------------------------------------------------

\begin{letter}{} 

%----------------------------------------------------------------------------------------
%	YOUR NAME & ADDRESS SECTION
%----------------------------------------------------------------------------------------

\begin{center}
\large\bf Hector Bahamonde \\ % Your name
Rutgers University\\
%\vspace{20pt} \hrule height 1pt % If you would like a horizontal line separating the name from the address, uncomment the line to the left of this text
89 George St. \\ New Brunswick, NJ 08854 \\ (732) 318-9650 \\ 
\href{mailto:hector.bahamonde@rutgers.edu}{Hector.Bahamonde@Rutgers.edu} \\
\href{http://www.hectorbahamonde.com}{www.HectorBahamonde.com}\\

{\huge\vspace{8mm} Research Statement}
\end{center} 

%\signature{\vspace{6cm}h.b., fall 2016} % Your name for the signature at the bottom

%----------------------------------------------------------------------------------------
%	LETTER CONTENT SECTION
%----------------------------------------------------------------------------------------
\opening{} 
 
Within comparative politics, my research explores the intersection between inequality and political development, from a political economy perspective. I use cutting-edge quantitative methods to analyze the role of inequality on state capacities, comparative historical analysis to trace the origins of institutional investments, and experimental methods to study individual preferences on vote-buying. I conduct research primarily in the context of Latin America, but also have projects on vote-selling and voting behavior in the U.S. Below, I summarize my job market paper on the origins of fiscal capacities, my book manuscript on the puzzle of state capacities in Latin America, and a series of working papers on vote-selling and vote-buying, both using observational and experimental data for the Americas.

% job market paper // motivation.
\emph{{\bf Job Market Paper} ``Structural Transformations and The Political Roots of Fiscal Capacities in Latin America''}. According  to many political economists, fiscal sociologists, development economist and economic historians, fiscal capacities are a prerequisite for ``strong'' states. Yet, the \emph{origins} of fiscal capacities are still unclear. There are widely accepted theories for the European case but there is not a consensus for the Latin American case. \emph{Why do some countries have ``better'' fiscal capacities than others? What have been the factors that led post-colonial Latin American countries to self-impose a system to directly tax individuals?} These questions are key to understand the development of the modern state in Latin America. In my job market paper, I inquire what were the \emph{conditions} that promoted the implementation of fiscal capacities in Latin America and trace back their origins.


% job market paper // argument.
My job market paper argues that the modernization of the fiscal apparatus was the product of a bargaining process between agricultural political incumbents and an emerging and politically excluded industrial sector. Different sectoral outputs leveraged each sector in a way that when industrial output was slow, agricultural incumbents would expropriate the modern sector. In turn, fast industrial output propitiated taxation and other institutional investments. Building on the fiscal sociology literature and the dual-sector economy model, I argue that these two approaches combined provide a new theoretical explanation for how industrial growth contributes to the formation of the ``main'' state capacity, in a way that differs entirely from modernization theory. 

% job market paper // methodology \\ research design // specific findings
To test this argument, I leverage sectoral outputs dating back to 1900 until 1980 of a sample of seven Latin American countries. Using panel data methods, I model the conditional hazard ratio that a country which has not yet adopted the income tax adopts in a given year as a function of the relative size of the agricultural and industrial sectors. I argue that income taxation triggered a series of other institutional investments, such as a strong party system and higher democratic levels. Within the multiple failure-time framework, I use a slightly different data structure and estimate the \emph{joint} occurrence of income taxation \emph{and} democratization. I find that as the size of the industrial sector grows, the likelihood of income taxation \emph{and} democratization grows exponentially. Using Chile as a case study, I observe what were the main dynamics between the two Chilean elites, confirming that the process of institutional investments comes from the dispute between them, and that the masses did not have that much of a role. 

% job market paper // methodology \\ research design // overall findings.
My findings suggest that industrial growth contributed to institutional investments and democratization by providing economic incentives for the elites to create equitably (between the two elites) administrative and fiscal state-powers. If institutions in Europe were designed to protect the bourgeoisie from expropriation-prone kings, institutions in Latin America were introduced to protect elites from themselves.

{\bf Book Manuscript} \emph{``Structural Transformations in Latin America: State Building and Elite Competition 1850-1980''}. My job market paper is part of my book manuscript project. By exploiting original data from a variety of perspectives, the book examines in a more profound manner how inter-elite disputes map into actual state capacities.

% first part: income tax paper. // % second part: dual economy.
The book manuscript is divided into four parts. The first part develops a theory for the origins of fiscal capacities, which is the topic of the job market paper. The second part argues that there exists a dialectic relationship between the expansion of the modern sector relative to the traditional sector. While the former grows and develops, the latter has to decay. This chapter argues that there are several structural factors that put restrictions on balanced growth of the two sectors. The most important of these is labor retention, and the eventual formation of the urban working-class. I claim that when agriculture productivity is low, there is abundant supply of cheap labor which the manufacturing sector can rely on. Even when productivity is high, and mechanized agricultural is adopted, it still frees labor to the modern sector. It is the modern sector that acts as the wheel of these economic structural transformations. I test this theory using time series econometric methods (vector error correction equations) and Granger causality tests to argue that the channel by which the industrial sector was expanded had to do with an initial consolidation of the agriculture sector followed by an even faster decay of it. 

% explanation of the first part
%Post-colonial Latin American polities were politically monopolized by elites invested in land or agriculture. Some factors led to the incipient development of an industrial sector which was consistently excluded from the political realm. The critical juncture that locked countries in underdevelopment traps occurred when initial monopolistic conditions were not broken. I claim that economically weak industrial sectors were not able to challenge agricultural elites, being unable to transform the old order. Under these conditions, there were not incentives to make institutional investments. However, when industrial output was fast, agricultural elites were able to force the modern sector to pay taxes. I develop a cost-benefit argument for why that is the case. Taxation forced both sectors to make political concessions. First, skilled and independent bureaucracies were needed to monitor and administer those taxes, and second, inter-elite electoral competition (in the form of an ``oligarchic democracy'') was needed to monitor inter-elite behavior. Via path dependency, these institutions got consolidated. I test this theory using Cox-Proportional hazard models and historical growth data from 1900 to 1970. 

% third part: state capacities.
The third part links these two structural transformations with actual state capacities. However, one of the biggest gaps in the literature is that there is not a consensus regarding how to measure state capacities. Another novel implication of my book manuscript is that it proposes measuring this concept using original earthquake data. The rationale behind this measurement is very intuitive, and yet has been systematically ignored in the literature. \emph{Why does a 7.0 earthquake flatten Haiti leaving at least 100,000 deaths while a 8.8 earthquake in Chile in the same year leaves just 525 deaths?} The answer is that in Chile, there are strong and heavily enforced regulations. Every building before being erected has to go through a very exhaustive approval process. Haiti lacks state capacities to implement and enforce such regulations. Good-quality building codes and zoning laws are a \emph{reflection} of other institutions too. Counting individuals (census), observing their incomes to tax them (fiscal capacities), policing them and protecting them from external menaces (monopoly of violence) and proportioning and enforcing an unitary legal code (monopoly of the rule of law), are all capacities that are nested in the same latent dimension, i.e. ``state capacities''. My project has received two grants that have allowed me to gather most of the data. By exploiting this historical data from 1900 to 2000 on earthquake death tolls and magnitudes for all southern cone Pacific coast Latin American countries, I provide evidence within states over time by estimating state-of-the-art spatial regression analyses to understand to which degree inter-elite conflicts (measured with sectoral output data) cause state capacities (proxied by death tolls associated to similar earthquakes). 

% fourth part: case studies and small-N section.
The fourth section of my book explores the micro-mechanisms of inter-elite disputes and state capacities within a small-N design. Here, I compare similar subnational contiguous units, for example towns in northern Chile and southern Peru, and observe in more detail the mechanics between inter-elite disputes and state-capacities when both units are affected by similar exogenous shocks (i.e. earthquakes). Keeping everything else constant, the difference in death tolls should be attributed to the degree in which the agriculture sector was challenged by the modern sector. To test this theory, I rely on process tracing, comparative historical analysis and archival analyses. 

% conclusion book manuscript
Overall, the dissertation urges a reevaluation of the origins of state capacities. This project is novel because it departs from the classical view that looks at external conflicts. Particularly, the project proposes that endogenous structural economic transformations shaped the incentives to build state institutions that helped to create states of high capacities. My research design is also novel as it exploits time variance within a large-N sample, going beyond the standard small-N historical-only comparison. Furthermore, it also develops a novel proxy for state capacities, which potentially can be expanded to other regions of the world.


% Vote-Buying in Brazil// Frame myself as a scholar of inequality 
{\bf Projects and Working Papers} As a scholar interested in Latin America, most of my academic work is centered around the issue of inequality. One manifestation of inequality that I am particularly invested in is the relationship between poverty, political (under)development and vote-buying, and clientelism and vote-selling. 

% Vote-Buying in Brazil// explaining the paper
{\bf Vote-Buying} I have a working paper on vote-buying in Brazil ready to submit for publication. The paper starts by recognizing that there is not an agreement on whether parties target groups or individuals. That is, there is not clarity on whether parties buy votes from groups or individuals. In fact, most of the times, scholars assume that group-targeting and individual-targeting are interchangeable. What seems to be a major problem, however, is that scholars seem to base their decision on their research designs; ethnographers typically study how parties target \emph{individuals} while experimentalist scholars typically look at how parities target districts/municipalities/states (i.e. \emph{groups}). In the paper, I develop and test a theory where parties make use of simultaneous segmented targeting techniques. Parties make their decision whether to target groups or individuals based on (1) the level of political competition, (2) the aggregated (i.e. municipal) and  (3) individual poverty levels. Groups are preferable by brokers when party machines need to secure higher levels of electoral support, relying on the economies of scale and spillover effects that these groups provide. However, individuals are better targets when they are more identifiable, that is when poor individuals are nested in non-poor contexts or when non-poor individuals are nested in poor groups.

{\bf Vote-Selling} Often times, political economists compare Latin America with the U.S. based on their different banking/financial systems, different colonial pasts and other relevant areas. Following this tradition, I received a grant to field a \emph{list} experiment and a \emph{conjoint} experiment to study vote-selling in the U.S. Not only did I study price elasticities, which are substantively important from a behavioral perspective, but also which of Dahl's democratic components\footnote{I.e., democratic, liberal and/or republican components.} should ``fail'' to make individuals more likely to sell their votes, which is a fundamental question in democratic theory and comparative politics. Currently, I am working on extending this project by including other Latin American countries. In addition to that, this is a novel project in two ways. First, clientelism scholars have often focused on vote-\emph{buying} (i.e. parties buying votes). Interestingly, we learn about parties' clientelistic behavior by asking voters via surveys. I propose the reverse, to study vote-\emph{selling} focusing on potential vote sellers. Second, behavioral scholars usually measure support for democracy using the standard question on this topic. However, this measurement does not capture properly ``true'' democratic support, as in Latin America, where this question is heavily confounded with left/right self-positioning. 

{\bf Summary and Future Research} In sum, my book manuscript, job market paper and working papers on vote-buying and experiments on vote-selling, work toward exploring the effects of inequality on political development from a comparative perspective. I use a widely broad methodological perspective, historical comparisons, time series analyses, and experimental and quasi-experimental methodologies. My goal is to use this toolkit to keep asking ``big'' questions that are fundamental for our discipline. My future research will seek to study the connection between elite competition and democratic regimes, exploring the connection between state building and democratic institutions and its relationship with vertical accountability and the role of the middle class on bureaucracy formation.

\closing{{\color{white}empty here}}


\encl{Curriculum vitae, teaching statement, cover letter, writing samples.}

%----------------------------------------------------------------------------------------

\end{letter}

\end{document}
