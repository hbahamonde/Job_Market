%----------------------------------------------------------------------------------------
%	PACKAGES AND OTHER DOCUMENT CONFIGURATIONS
%----------------------------------------------------------------------------------------

\documentclass[11pt]{letter} % Default font size of the document, change to 10pt to fit more text

%\usepackage{newcent} % Default font is the New Century Schoolbook PostScript font 
%\usepackage{helvet} % Uncomment this (while commenting the above line) to use the Helvetica font

% packages
\usepackage{hyperref} % for links
\usepackage[bottom]{footmisc} % footnotes at the bottom.
\usepackage{color} % for colors

% Margins
\topmargin=-1in % Moves the top of the document 1 inch above the default
\textheight=9in % Total height of the text on the page before text goes on to the next page, this can be increased in a longer letter
\oddsidemargin=-10pt % Position of the left margin, can be negative or positive if you want more or less room
\textwidth=6.5in % Total width of the text, increase this if the left margin was decreased and vice-versa

%\let\raggedleft\raggedright % Pushes the date (at the top) to the left, comment this line to have the date on the right
\date{}


\usepackage{color} % for colors
\usepackage{hyperref} % for links

\hypersetup{
  colorlinks = true,
  urlcolor = blue,
  pdfpagemode = UseNone
}



\begin{document}

%----------------------------------------------------------------------------------------
%	ADDRESSEE SECTION
%----------------------------------------------------------------------------------------

\begin{letter}{} 

%----------------------------------------------------------------------------------------
%	YOUR NAME & ADDRESS SECTION
%----------------------------------------------------------------------------------------

\begin{center}
\large\bf Hector Bahamonde \\ % Your name
Rutgers University\\
\vspace{20pt} \hrule height 1pt % If you would like a horizontal line separating the name from the address, uncomment the line to the left of this text
89 George St. \\ New Brunswick, NJ 08901 \\ (732) 318-9650 \\ 
{\normalfont\normalsize\href{mailto:hector.bahamonde@rutgers.edu}{Hector.Bahamonde@Rutgers.edu}} \\
{\normalfont\normalsize\href{http://www.hectorbahamonde.com}{www.HectorBahamonde.com}}\\
{\normalfont \scriptsize{
% Date
\vspace{5mm}\today\\
Download last version \href{http://github.com/hbahamonde/Job_Market/raw/master/Bahamonde_Research_Statement.pdf}{\texttt{\color{red}here}}}} % Link to last version
\\
{\huge\vspace{3mm} Research Statement}
\end{center} 

%\signature{\vspace{6cm}h.b., 2017} % Your name for the signature at the bottom

%----------------------------------------------------------------------------------------
%	LETTER CONTENT SECTION
%----------------------------------------------------------------------------------------
\opening{} 
 
% intro // job market paper
\vspace{-1.5cm}Within comparative politics, my research explores the intersection between inequality and political development, from a political economy perspective. My {\bf job market paper} (\emph{under review}) entitled \href{https://github.com/hbahamonde/IncomeTaxAdoption/raw/master/Bahamonde_IncomeTaxAdoption.pdf}{``Sectoral Origins of Income Taxation: Industrial Development and The Case of Chile (1900-2010)''} starts from the observation that an elite divided on an economic cleavage is at the same time divided on their political preferences, particularly regarding their attitude towards state centralization. Consequently, an elite split along economic interests will use state power to influence certain policies and hence, growth and state building. The paper argues that the modernization of the fiscal apparatus was product of an inter-sectoral conflict between the agricultural and industrial elites. 

% main argument of the book
The job market paper is embedded into a larger {\bf book manuscript} entitled \emph{``Structural Transformations in Latin America: State Building and Elite Competition 1850-2010''} where I analyze how a major change in the economy acted as a critical juncture setting countries in either a path of development or a development trap. The {\bf main argument of the book} is that the economic structural transformation in Latin America, i.e. the secular decline of agriculture and substantial expansion of manufacturing, imposed tight constraints on the way politics was run by the incumbent landowning class. The emergence of an efficient and productive industrial sector altered not only the structure of the economy (causing growth) but also the inter-sectoral balance of political power, making unsustainable the political monopoly run by the landed elites inherited from the colonial period. The landowning classes in Latin America had been a hegemonic group protected by institutions that originated in colonial times. These norms had survived due to institutional inertia, perpetuating their advantaged position. Following the dual sector economy model, the book argues that the structural transformation required both sectors to grow in a \emph{balanced} fashion, \emph{leveling both elites in their relative political and military capacities}. This argument differs deeply from modernization theory. What causes political development is not industrialization \emph{per se}, but the development of a productive landed elite which supplied labor and cheap foodstuff which the modern sector could rely on, demanding these goods, promoting that way balanced economic development between the two sectors, \emph{politically empowering both economic elites}. Leveraging sectoral outputs dating back to 1900 until 2010 of a sample of Latin American countries, the book uses panel data methods (particularly Cox-proportional hazard models), time series analyses (VAR models, impulse response functions and Granger-causality tests) and fine-grained qualitative data to support its argument. The paper can be found \href{https://github.com/hbahamonde/Negative_Link_Paper/raw/master/Bahamonde_NegativeLink.pdf}{\texttt{here}}.

% measurement
Another mayor contribution of the book is measurement. One of the biggest gaps in the literature is the lack of a measurement of state capacities able to capture variations through time. Using a novel dataset, the book proposes measuring state capacities using earthquake data. The rationale is very intuitive: good-quality building codes and zoning laws are a \emph{reflection} of other \emph{state institutions}. \emph{Why does a 7.0 earthquake flatten Haiti leaving at least 100,000 deaths while a 8.8 earthquake in Chile in the same year leaves just 525 deaths?} In Chile, there are strong and heavily enforced regulations. Every building before being erected has to go through a very exhaustive approval process. \emph{Haiti lacks state capacities to implement and enforce such regulations}. By exploiting historical variation going back to 1900 until 2010 on earthquake death tolls (and local population to weight the number of deaths) I measure state capacities through time. Keeping magnitudes constant, any difference in the number of deaths associated to earthquakes should be attributed to the \emph{lack of state capacities}. 

% Vote-Buying in Brazil// explaining the paper
{\bf Vote-buying.} Going forward, I have a {\bf working \href{https://github.com/hbahamonde/Clientelism_paper/raw/master/Bahamonde_Clientelism_Paper.pdf}{paper}} on vote-buying in Brazil. The paper starts by recognizing that there is not an agreement on whether parties target groups or individuals. In fact, most of the times, scholars assume that group-targeting and individual-targeting are interchangeable. The paper argues that parties make use of simultaneous segmented targeting techniques. Groups are preferable by brokers when party machines need to secure higher levels of electoral support, relying on economies of scale and spillover effects that these groups provide. However, individuals are better targets when they are more identifiable, that is when poor individuals are nested in non-poor contexts or when non-poor individuals are nested in poor groups. The theory also explains why non-poor individuals are also targeted. The paper uses observational data and matching methods and a short case study (Brazil) to confirm the assumptions made in the models.

% vote-selling // experiments
{\bf Vote-Selling.} With the support of a generous grant, I designed two {\bf experiments} in the U.S. out of a series of experiments to be fielded in Latin America for further comparison. I look at the tipping points where U.S. citizens prefer a monetary incentive rather than keeping their right to choose for whom to vote for. My identification strategy takes advantage of a \emph{list experiment} to capture non-biased answers on socially-condemnable/illegal behaviors (like vote-buying). In a separate study, I designed a \emph{conjoint experiment} to identify which of Dahl's democratic dimensions should fail to predict individual propensities of vote-selling. Conjoint experiments allow researchers to directly isolate complex multi-dimensional concepts (such as \emph{support for democracy}) and observe which dimension(s) is/are associated to the outcome of interest (vote-selling). Preliminary results show that when the \emph{liberal} component, particularly the \emph{right to associate} fails, individuals are more likely to sell. In this sense, this project is very innovative as it departs from the common strategy of correlating the standard support for democracy question with vote-buying. My identification strategy is more comprehensive as it \emph{decomposes} ``support for democracy'' in several dimensions which are theoretically and substantively relevant (i.e., Dahl's democracy dimensions). 


{\bf Summary and Future Research} In sum, my book manuscript, job market paper and working papers on vote-buying and experiments on vote-selling, work toward exploring the effects of inequality on political development from a comparative perspective. I use a widely broad methodological perspective, historical comparisons, time series analyses, and experimental and quasi-experimental methodologies. My goal is to use this toolkit to keep asking ``big'' questions that are fundamental for our discipline. My future research will seek to study the connection between elite competition and democratic regimes, exploring the connection between state building and democratic institutions and its relationship with vertical accountability and the role of the middle class on bureaucracy formation.\vspace{-10cm}

\closing{{\color{white}empty here}}


{\color{white}\encl{}}

%----------------------------------------------------------------------------------------

\end{letter}

\end{document}
