\RequirePackage{atbegshi}
\documentclass[compress,dvipsnames, aspectratio=169]{beamer} % aspectratio=169
%\usepackage[svgnames]{xcolor}

%	%	%	%	%	%	%	%	%	%	%	%	%	%	%
% 						MY PACKAGES 
%	%	%	%	%	%	%	%	%	%	%	%	%	%	%
\usepackage{graphicx}				% Use pdf, png, jpg, or eps with pdflatex; use eps in DVI mode

\usepackage[export]{adjustbox}

\usepackage{amssymb}
\usepackage{amsmath}	
%\usepackage{tipx}
%\usepackage{tikz}
%\usetikzlibrary{arrows,shapes,decorations.pathmorphing,backgrounds,positioning,fit,petri}
\usepackage{rotating}
\usepackage{scalerel} % for inline images
\usepackage{import}
%\usepackage{times}
\usepackage{array}
\usepackage{tabularx}
%\usepackage{booktabs}
\usepackage{textcomp}
\usepackage{caption}
\usepackage{float}
%\usepackage{setspace} 			% \doublespacing \singlespacing \onehalfspacing	%doble espacio
%\label{x:y}													%ocupar para autoref.
%\autoref{x:y}												%ocupar para autoref.
%\usepackage{nopageno}			%desactivar para p�ginas
\usepackage{pifont}
\usepackage{color,xcolor,ucs}
%\usepackage{marvosym} %faces

\usepackage{hyperref}
\usepackage{multirow}

\usepackage{listings}
\usepackage{color}
\definecolor{dkgreen}{rgb}{0,0.6,0}
\definecolor{gray}{rgb}{0.5,0.5,0.5}
\definecolor{mauve}{rgb}{0.58,0,0.82}
\lstset{ %
  language=R,                     % the language of the code
  basicstyle=\TINY,      			% the size of the fonts that are used for the code
  numbers=left,                   % where to put the line-numbers
  numberstyle=\tiny\color{gray},  % the style that is used for the line-numbers
  stepnumber=1,                   % the step between two line-numbers. If it's 1, each line
                                  % will be numbered
  numbersep=5pt,                  % how far the line-numbers are from the code
  backgroundcolor=\color{white},  % choose the background color. You must add \usepackage{color}
  showspaces=false,               % show spaces adding particular underscores
  showstringspaces=false,         % underline spaces within strings
  showtabs=false,                 % show tabs within strings adding particular underscores
  frame=single,                   % adds a frame around the code
  rulecolor=\color{black},        % if not set, the frame-color may be changed on line-breaks within not-black text (e.g. commens (green here))
  tabsize=1,                      % sets default tabsize to 2 spaces
  captionpos=b,                   % sets the caption-position to bottom
  breaklines=true,                % sets automatic line breaking
  breakatwhitespace=false,        % sets if automatic breaks should only happen at whitespace
  title=\lstname,                 % show the filename of files included with \lstinputlisting;
                                  % also try caption instead of title
  keywordstyle=\color{blue},      % keyword style
  commentstyle=\color{dkgreen},   % comment style
  stringstyle=\color{mauve},      % string literal style
  escapeinside={\%*}{*)},         % if you want to add a comment within your code
  morekeywords={*,...}            % if you want to add more keywords to the set
} 

%	%	%	%	%	%	%	%	%	%	%	%	%	%	%
% 					PACKAGE CUSTOMIZATION
%	%	%	%	%	%	%	%	%	%	%	%	%	%	%

% GENERAL CUSTOMIZATION
\usepackage[math]{iwona}% font
\usetheme{Singapore}	% template I should use
%\usetheme{Szeged}	% alternative template
\usecolortheme{rose}	% color template
\makeatletter			% to show subsection/section title (1/3)
\beamer@theme@subsectiontrue % to show subsection/section title (2/3)
\makeatother			% to show subsection/section title (3/3)



% THIS BELOW IS TO MAKE NAVIGATION DOTS MARKED DURING PRESENTATION
\makeatletter
\def\slideentry#1#2#3#4#5#6{%
  %section number, subsection number, slide number, first/last frame, page number, part number
  \ifnum#6=\c@part\ifnum#2>0\ifnum#3>0%
    \ifbeamer@compress%
      \advance\beamer@xpos by1\relax%
    \else%
      \beamer@xpos=#3\relax%
      \beamer@ypos=#2\relax%
    \fi%
  \hbox to 0pt{%
    \beamer@tempdim=-\beamer@vboxoffset%
    \advance\beamer@tempdim by-\beamer@boxsize%
    \multiply\beamer@tempdim by\beamer@ypos%
    \advance\beamer@tempdim by -.05cm%
    \raise\beamer@tempdim\hbox{%
      \beamer@tempdim=\beamer@boxsize%
      \multiply\beamer@tempdim by\beamer@xpos%
      \advance\beamer@tempdim by -\beamer@boxsize%
      \advance\beamer@tempdim by 1pt%
      \kern\beamer@tempdim
      \global\beamer@section@min@dim\beamer@tempdim
      \hbox{\beamer@link(#4){%
          \usebeamerfont{mini frame}%
          \ifnum\c@section>#1%
            %\usebeamercolor[fg]{mini frame}%
            %\usebeamertemplate{mini frame}%
            \usebeamercolor{mini frame}%
            \usebeamertemplate{mini frame in other subsection}%
          \else%
            \ifnum\c@section=#1%
              \ifnum\c@subsection>#2%
                \usebeamercolor[fg]{mini frame}%
                \usebeamertemplate{mini frame}%
              \else%
                \ifnum\c@subsection=#2%
                  \usebeamercolor[fg]{mini frame}%
                  \ifnum\c@subsectionslide<#3%
                    \usebeamertemplate{mini frame in current subsection}%
                  \else%
                    \usebeamertemplate{mini frame}%
                  \fi%
                \else%
                  \usebeamercolor{mini frame}%
                  \usebeamertemplate{mini frame in other subsection}%
                \fi%
              \fi%
            \else%
              \usebeamercolor{mini frame}%
              \usebeamertemplate{mini frame in other subsection}%
            \fi%
          \fi%
        }}}\hskip-10cm plus 1fil%
  }\fi\fi%
  \else%
  \fakeslideentry{#1}{#2}{#3}{#4}{#5}{#6}%
  \fi\ignorespaces
  }
\makeatother

%	%	%	%	%	%	%	%	%	%	%	%	%	%	%
% 			To show the TITLE at the Bottom of each slide
%	%	%	%	%	%	%	%	%	%	%	%	%	%	%

\beamertemplatenavigationsymbolsempty 
\makeatletter
\setbeamertemplate{footline}
{
\leavevmode%
\hbox{%
\begin{beamercolorbox}[wd=1\paperwidth,ht=2.25ex,dp=2ex,center]{title in head/foot}%
\usebeamerfont{title in head/foot}\insertshorttitle
\end{beamercolorbox}%
\begin{beamercolorbox}[wd=1
\paperwidth,ht=2.25ex,dp=2ex,center]{date in head/foot}%
\end{beamercolorbox}}%
}
\makeatother



% to switch off navigation bullets
%% using \miniframeson or \miniframesoff
\makeatletter
\let\beamer@writeslidentry@miniframeson=\beamer@writeslidentry
\def\beamer@writeslidentry@miniframesoff{%
  \expandafter\beamer@ifempty\expandafter{\beamer@framestartpage}{}% does not happen normally
  {%else
    % removed \addtocontents commands
    \clearpage\beamer@notesactions%
  }
}
\newcommand*{\miniframeson}{\let\beamer@writeslidentry=\beamer@writeslidentry@miniframeson}
\newcommand*{\miniframesoff}{\let\beamer@writeslidentry=\beamer@writeslidentry@miniframesoff}
\makeatother

% Image full size: use 
%%\begin{frame}
  %%\fullsizegraphic{monogram.jpg}
%%\end{frame}
\newcommand<>{\fullsizegraphic}[1]{
  \begin{textblock*}{0cm}(-1cm,-3.78cm)
  \includegraphics[width=\paperwidth]{#1}
  \end{textblock*}
}


% hyperlinks
\hypersetup{colorlinks,
            urlcolor=[rgb]{0.01, 0.28, 1.0},
            linkcolor=[rgb]{0.01, 0.28, 1.0}}








%	%	%	%	%	%	%	%	%	%	%	%	%	%	%
% 					DOCUMENT ID
%	%	%	%	%	%	%	%	%	%	%	%	%	%	%

\title{Interview with Turku University}
\author{Hector Bahamonde $\bullet$ Assistant Professor $\bullet$ O'Higgins University (Chile)}
\date{June 22nd, 2021}

%to to see shadows of previous blocks
%\setbeamercovered{dynamic}


\begin{document}



%	%	%	%	%	%	%	%	%	%	%	%	%	%	%
% 					CONTENT
%	%	%	%	%	%	%	%	%	%	%	%	%	%	%

%% title frame

\begin{frame}[label = cover]
\titlepage
\end{frame}



\section{Introduction}

%section{Outline}
\subsection{Introduction}

\miniframesoff
\begin{frame}\frametitle{The Order of the Day}
In this presentation I will...
\begin{enumerate}
	\item Briefly describe my {\bf profile}.
	\item Quickly mention three of my most important {\bf publications}.
	\item Explain my 3-year {\bf research plans} at Turku.
	\item Enumerate the three main reasons to {\bf move to Turku}.
\end{enumerate}
\end{frame}


\subsection{Short Bio}

\miniframeson
\begin{frame}\frametitle{Short Bio}
I am a political scientist (B.A. and PhD)...
\begin{itemize}
	\item Whose primary subfield is the political economy of {\bf inequality}, {\bf political development} and {\bf clientelism}. 
	\item Who has a strong interest in statistical and experimental methods (natural, lab and survey-based).
	\item Who is currently an assistant professor in Chile---but looking forward to relocate to Europe; family reasons. {\bf Immediate availability}.
\end{itemize}
\end{frame}

\subsection{Published Work}


%\miniframeson
%\begin{frame}\frametitle{Main Published Work}
%	\begin{enumerate}
%		\item ``{\bf Inclusive Institutions, Unequal Outcomes: Democracy, State Capacity, and Income Inequality}.''
%			\begin{itemize}
%				\item {\color{ForestGreen}Democratic theory, inequality and stateness}.
%				\item Time-series and fixed-effects methods.
%				\item {\color{blue}European Journal of Political Economy} (forthcoming).
%			\end{itemize}
%		\item ``{\bf Still for Sale: The Micro-Dynamics of Vote Selling in the United States, Evidence From a List Experiment}.''
%			\begin{itemize}
%				\item {\color{ForestGreen}Democratic development and clientelism}.
%				\item Survey experiment (list experiment) implemented via \texttt{Qualtrics}.
%				\item \emph{Original} data representative at the U.S. level.
%				\item {\color{blue}Acta Politica} (forthcoming).
%			\end{itemize}
%		\item ``{\bf Aiming Right at You: Group versus Individual Clientelistic Targeting in Brazil}.''
%			\begin{itemize}
%				\item {\color{ForestGreen}Inequality and clientelism}. 
%				\item Observational data and matching methods for causal inference.
%				\item {\color{blue}Journal of Politics in Latin America} (2018).
%			\end{itemize}
%	\end{enumerate}
%\end{frame}


\miniframeson
\begin{frame}\frametitle{``{\bf Inclusive Institutions, Unequal Outcomes: Democracy, State Capacity, and Income Inequality}.''}
	\begin{columns}
	\column{0.5\textwidth}
			\begin{itemize}
				\item {\color{ForestGreen}Democratic theory, inequality and stateness}.
				\item 126 industrial and developing countries, between 1970 and 2013 (N=4,000).
				\item Time-series and fixed-effects methods.
				\item European Journal of Political Economy (forthcoming).
			\end{itemize}

	\column{0.5\textwidth}
		\begin{figure}[H]
		%\vspace{-2.5cm}
		\hspace{-7mm}
		\includegraphics[scale=0.2]{/Users/hectorbahamonde/Job_Market/cover_ejpe.jpg}
		\end{figure}
	
	\end{columns}
\end{frame}


\miniframesoff
\begin{frame}\frametitle{``{\bf Still for Sale: The Micro-Dynamics of Vote Selling in the United States, Evidence From a List Experiment}.''}
	\begin{columns}
	\column{0.5\textwidth}
			\begin{itemize}
				\item {\color{ForestGreen}Democratic development and clientelism}.
				\item I designed a survey experiment (list experiment) and implemented it in \texttt{Qualtrics}.
				\item \emph{Original} data representative at the U.S. level.
				\item Acta Politica (forthcoming).
			\end{itemize}

	\column{0.5\textwidth}
		\begin{figure}[H]
		%\vspace{-2.5cm}
		\hspace{-7mm}
		\includegraphics[scale=0.3]{/Users/hectorbahamonde/Job_Market/cover_ap.jpeg}
		\end{figure}
	\end{columns}
\end{frame}


\miniframesoff
\begin{frame}\frametitle{``{\bf Aiming Right at You: Group versus Individual Clientelistic Targeting in Brazil}.''}
	\begin{columns}
	\column{0.5\textwidth}
			\begin{itemize}
				\item {\color{ForestGreen}Inequality and clientelism}. 
				\item Observational data and matching methods for causal inference.
				\item Journal of Politics in Latin America (2018).
			\end{itemize}

	\column{0.5\textwidth}
		\begin{figure}[H]
		%\vspace{-2.5cm}
		\hspace{-7mm}
		\includegraphics[scale=0.2]{/Users/hectorbahamonde/Job_Market/cover_jpla.png}
		\end{figure}
	\end{columns}
\end{frame}





\section{Research Plans}

\subsection{Pipeline summary}

\miniframesoff
\begin{frame}
\centering
\huge{My research plan seeks to study these issues by moving forward several pieces of research I have in the pipeline.}
\end{frame}

% HERE


\miniframeson
\begin{frame}\frametitle{Pipeline 1: Clientelism}
\begin{enumerate} \setcounter{enumi}{0}
\item {\bf Lab and survey experiments}.
	\begin{itemize}
		\item ``{\input{/Users/hectorbahamonde/research/Economic_Experiment_Vote_Selling/title.txt}\unskip}'' ({\color{blue}{\input{/Users/hectorbahamonde/research/Economic_Experiment_Vote_Selling/status.txt}\unskip}}).
		
		\item ``{\input{/Users/hectorbahamonde/research/Conjoint_US/title_letter.txt}\unskip}'' ({\color{blue}{\input{/Users/hectorbahamonde/research/Conjoint_US/status_letter.txt}\unskip}}).
	\end{itemize}
\end{enumerate}
\end{frame}

\miniframesoff
\begin{frame}\frametitle{Pipeline 2: Inequality and COVID19}
\begin{enumerate} \setcounter{enumi}{1}
\item {\bf Natural experiments}.
	\begin{itemize}
		\item ``{\input{/Users/hectorbahamonde/research/Tobalaba/title.txt}\unskip}'' ({\color{blue}{\input{/Users/hectorbahamonde/research/Tobalaba/status.txt}\unskip}}).
		
		\item ``{\input{/Users/hectorbahamonde/research/Bus/title.txt}\unskip}'' ({\color{blue}{\input{/Users/hectorbahamonde/research/Bus/status.txt}\unskip}}).

	\end{itemize}
\end{enumerate}
\end{frame}

\subsection{Pipeline: survey and lab experiments}

\miniframeson
\begin{frame}\frametitle{{\scriptsize{``\input{/Users/hectorbahamonde/research/Economic_Experiment_Vote_Selling/title.txt}\unskip.}''}}
	%\begin{columns}
	%\column{0.5\textwidth}
		\begin{itemize}
			\item {\color{blue}Economic experiments are ``games''}: subjects embody randomly-assigned roles {\scriptsize(2 parties/1 voter)}.

			\item {\color{ForestGreen}Democratic theory}: study conditions of {\bf (i)lliberal democracy} that foster clientelism:
						\begin{enumerate}
							\item {\bf Endowments}: different for ``parties'' and ``voters'' {\scriptsize({\color{blue}emulates income inequality})}.
							\item {\bf Ideology}: ideological distance between ``parties'' and ``voters'' {\scriptsize({\color{blue}emulates issue/spatial location})}.
							\item {\bf Contestation}: ``Risk'' of losing the election {\scriptsize({\color{blue}emulates party competition})}.
						\end{enumerate}
		\item {\scriptsize{\bf Data are being collected as we speak} (N=200).}
		\end{itemize}


	%\column{0.5\textwidth}
		%\begin{figure}[H]
		%\vspace{-2.5cm}
		%\hspace{-7mm}


		%\includegraphics[scale=0.5]{/Users/hectorbahamonde/research/Economic_Experiment_Vote_Selling/Experimental_Flow_Figure.pdf}
		%\end{figure}
	
	%\end{columns}
\vspace{\fill}\hspace{\fill}\hyperlink{econ_exp}{\beamerbutton{Design}}
\end{frame}


\miniframeson
\begin{frame}\frametitle{{\scriptsize{``{\input{/Users/hectorbahamonde/research/Conjoint_US/title_letter.txt}\unskip}'' ({\color{blue}{\input{/Users/hectorbahamonde/research/Conjoint_US/status_letter.txt}\unskip}}).}}}	
\begin{columns}
	\column{0.5\textwidth}	
		\begin{itemize}
			\item I designed a {\color{blue}conjoint experiment} in \texttt{Qualtrics}.
			\item Conjoint experiments are {\color{blue}{good to study the causal effect of multiple-attribute treatments}}.
			\item {\color{ForestGreen}Democratic theory}: Using Dahl's Polyarchy (1971), I devised different ``political candidates'' who supported different ``policies.''
			\item {\scriptsize Tasked experimental subjects with choosing a candidate.}
		\end{itemize}
		\column{0.5\textwidth}	
\begin{table}[h]
\begin{center}
{\renewcommand{\arraystretch}{2}%
\scalebox{0.4}{
\hspace{-2cm}
\begin{tabular}{cc}
\hline
\multicolumn{1}{|c|}{{\bf Candidate 1}}   & \multicolumn{1}{c|}{{\bf Candidate 2}}  \\ \hline
\multicolumn{1}{|c|}{\small{Media CAN confront the government}}    & \multicolumn{1}{c|}{\small{Media CANNOT confront the government}}   \\ \hline
\multicolumn{1}{|c|}{\small{President CANNOT rule without Congress}}    & \multicolumn{1}{c|}{\small{President CAN rule without Congress}}   \\ \hline
\multicolumn{1}{|c|}{\small{Citizens CANNOT vote in the next two elections}}    & \multicolumn{1}{c|}{\small{Citizens CANNOT vote in the next two elections}}   \\ \hline
\multicolumn{1}{|c|}{\small{Citizens CAN run for office for the next two elections}}    & \multicolumn{1}{c|}{\small{Citizens CAN run for office for the next two elections}}   \\ \hline
\multicolumn{1}{|c|}{\small{Citizens CAN associate with others and form groups}}    & \multicolumn{1}{c|}{\small{Citizens CANNOT associate with others and form groups}}   \\ \hline
\multicolumn{2}{c}{\texttt{Which of these candidates represents the lesser of the two evils for you?}} \\ \hline
\multicolumn{1}{|c|}{\texttt{Candidate 1} {\large$\square$}} & \multicolumn{1}{c|}{\texttt{Candidate 2} {\large$\square$}} \\ \hline
\end{tabular}}
}
\end{center}
\caption*{{\tiny Attributes are assigned at random. This is just one realization. Five of these tasks were administered for every subject.}}
\end{table}
\end{columns}
\vspace{\fill}\hspace{\fill}\hyperlink{conjoint_table}{\beamerbutton{Details}}
\end{frame}


\miniframesoff
\begin{frame}\frametitle{{\scriptsize{``{\input{/Users/hectorbahamonde/research/Conjoint_US/title_letter.txt}\unskip}'' ({\color{blue}{\input{/Users/hectorbahamonde/research/Conjoint_US/status_letter.txt}\unskip}}).}}}	

\begin{columns}
	\column{0.65\textwidth}	


		\begin{itemize}
			\item {\color{ForestGreen}Innovative way of exploiting conjoint experimental data}: 
				\begin{enumerate}
					\item Using {\bf machine learning methods}, we exploit those data to {\bf classify likely vote-sellers}.
					\item Building on Dahl (1971), {\color{blue}subjects who distrust ``Free media'' and ``Presidential dependence'' (on Congress) are more likely to sell their vote} in the U.S.
				\end{enumerate}
			\item {\scriptsize{\bf I already have this novel data---representative at the United States level} (N=1,108).}
		\end{itemize}

	\column{0.5\textwidth}
		\begin{figure}[H]
		\vspace{-0.5cm}
		\hspace{-4mm}
		\includegraphics[scale=0.4]{/Users/hectorbahamonde/research/Conjoint_US/plot_vote_selling.pdf}
		\end{figure}



\end{columns}
\vspace{\fill}\hspace{\fill}\hyperlink{conjoint_table}{\beamerbutton{Details}}
\end{frame}




\subsection{Pipeline 2: Inequality and COVID19}

\miniframeson
\begin{frame}\frametitle{{\scriptsize{``{\input{/Users/hectorbahamonde/research/Tobalaba/title.txt}\unskip}.''}}}		

\begin{columns}
\column{0.6\textwidth}

\begin{itemize}
	\item {\scriptsize\emph{Santiago de Chile} is the capital city of one of the most unequal countries in the world.}
\end{itemize}

\begin{enumerate}\setcounter{enumi}{0}
	\item {\color{ForestGreen}{{\bf Unequal} Application of the Rule of Law}}: 
		\begin{itemize}
			\item While the state was able to control ordinary citizens when traveling, it ``failed'' to control airspace.
			\item Study how elites were able to travel to their vacations houses during lockdown via a small aerodrome located in one of the richest municipalities. 
		\end{itemize}
\end{enumerate}


\column{0.4\textwidth}
		\begin{figure}[c]
		\vspace{-0.1cm}\hspace{-7mm}\includegraphics[scale=0.15]{/Users/hectorbahamonde/research/Tobalaba/control.jpg}
		\\
		\vspace{0.5cm}{\hspace{-8mm}\includegraphics[scale=0.23]{/Users/hectorbahamonde/research/Tobalaba/ts_airport.pdf}}
		\end{figure}


\end{columns}

	%\item {\color{ForestGreen}{Natural experiment}}:
				%\begin{itemize}
					%\item {\bf Identification strategy}: The aerodrome is strictly used by the elite.
					%\item {\bf RDD}: confinement policies are exogenous and not-random.
					%\emph{How effective were the lockdown policies {\bf for the elite}}? % dos series de tiempo  
				%\end{itemize}
%\item[] {\scriptsize{\color{blue}\emph{Were elites effectively escaping confinement policies and most importantly fast contagion rates}?}} % dos series de tiempo  


\vspace{\fill}\hspace{\fill}\hyperlink{tobalaba}{\beamerbutton{Details}}
\end{frame}


\miniframesoff
\begin{frame}\frametitle{{\scriptsize{``{\input{/Users/hectorbahamonde/research/Tobalaba/title.txt}\unskip}.''}}}		

\begin{columns}
\column{0.5\textwidth}

\begin{enumerate}\setcounter{enumi}{1}
	\item {\color{ForestGreen}{Natural experiment}}:
				\begin{itemize}
					\item {\bf Identification strategy}: The aerodrome is \emph{strictly} used by the elite.
					\item {\bf RDD}: confinement policies are exogenous and not-random.
					\emph{How effective were the lockdown policies {\bf for the elite}}? % dos series de tiempo  
				\end{itemize}
\end{enumerate}

	\column{0.5\textwidth}
		\begin{figure}[H]
		\vspace{-1cm}
		\hspace{-7mm}
		\includegraphics[scale=0.25]{/Users/hectorbahamonde/research/Tobalaba/map.pdf}
		\end{figure}

\end{columns}
\vspace{\fill}\hspace{\fill}\hyperlink{tobalaba}{\beamerbutton{Details}}
\end{frame}


\miniframeson
\begin{frame}\frametitle{{\scriptsize{``{\input{/Users/hectorbahamonde/research/Bus/title.txt}\unskip}.''}}}	

\begin{columns}
\column{0.5\textwidth}
{\color{ForestGreen}{Welfare, Covid and Inequality}}: 

		\begin{itemize}
			\item Chilean social safety net is very thin. 
			\item While the wealthy were able to work from home (or flight to their vacation houses), the working class kept riding the bus to work presentially. 
			%\item This paper explores a digitalized population dataset on daily public transportation and contagions.
			%\item Hypothesis is that the poor bore the bore the cost of the COVID pandemic.\\
			%Lower death thresholds for implementing total lockdown in wealthy municipalities.
		\end{itemize}

\column{0.5\textwidth}
		\begin{figure}[c]
			\vspace{-0.1cm}\hspace{-7mm}\includegraphics[scale=0.3]{/Users/hectorbahamonde/research/Bus/BusPlot.pdf}
		\end{figure}
\end{columns}
\vspace{\fill}\hspace{\fill}\hyperlink{tobalaba}{\beamerbutton{Details}}
\end{frame}


\miniframesoff
\begin{frame}\frametitle{{\scriptsize{``{\input{/Users/hectorbahamonde/research/Bus/title.txt}\unskip}.''}}}	

\begin{columns}
\column{0.5\textwidth}
{\color{ForestGreen}{Welfare, Covid and Inequality}}: 

		\begin{itemize}
			%\item Chilean social safety net is very thin. 
			%\item While the wealthy were able to work from home (or flight to their vacation houses), the working class kept riding the bus to work presentially. 
			\item This paper explores a digitalized population dataset on daily public transportation and contagions.
			\item Hypothesis is that the poor bore the bore the cost of the COVID pandemic.\\
			Lower death thresholds for implementing total lockdown in wealthy municipalities.
		\end{itemize}

\column{0.5\textwidth}
		\begin{figure}[c]
			\vspace{-0.1cm}\hspace{-7mm}\includegraphics[scale=0.3]{/Users/hectorbahamonde/research/Bus/BusPlot.pdf}
		\end{figure}
\end{columns}
\vspace{\fill}\hspace{\fill}\hyperlink{tobalaba}{\beamerbutton{Details}}
\end{frame}




\section{Last But Not Least}

\begin{frame}\frametitle{Last But Not Least}
\begin{itemize}
	\item Attending the main conferences in the discipline.
	\item Organizing a workshop/mini-conference per year at your institution.
	\item Giving service to the Philosophy, Contemporary History and Political Science Department, particularly, political science unit.
	\item Teaching and/or advising undergraduate/graduate courses/students.
	\item Assuming administrative tasks when necessary.
\end{itemize}
\end{frame}




\section{Appendix}

\subsection{Pipeline: Explained}


\miniframesoff
\begin{frame}\frametitle{{\scriptsize{``\input{/Users/hectorbahamonde/research/Economic_Experiment_Vote_Selling/title.txt}\unskip.}''}}\label{econ_exp}
\begin{columns}

\column{0.5\textwidth}
{\color{blue}{\bf Tell a supply and demand story}:}
\begin{itemize}
	\item Do {\color{orange}parties} target {\bf own supporters} {\scriptsize(Dixit/Londregan and Cox/McCubbins)} or {\bf moderate opposer} {\scriptsize(Stokes)}? 
	\item At what price?
	\item Under what conditions do {\color{green}vote sellers} sell to their own party of choosing?
\end{itemize}

\column{0.5\textwidth}
	\begin{figure}[H]
		\vspace{-1cm}
		\hspace{-7mm}
		\includegraphics[scale=0.5]{/Users/hectorbahamonde/research/Economic_Experiment_Vote_Selling/Experimental_Flow_Figure.pdf}
	\end{figure}
\end{columns}
\end{frame}


\miniframesoff
\begin{frame}\frametitle{{\scriptsize{``{\input{/Users/hectorbahamonde/research/Conjoint_US/title_letter.txt}\unskip}'' ({\color{blue}{\input{/Users/hectorbahamonde/research/Conjoint_US/status_letter.txt}\unskip}}).}}\label{conjoint_table}}
	\begin{columns}\column{0.7\textwidth}
		\begin{itemize}
			\item {\color{blue}Democratic theory}: Dahl (1971) specifies a number of dimensions any ``polyarchy'' should accomplish (\emph{free press}, \emph{free competition}, \emph{right to run for elections}, etc.)
			\item The experiment captures in a causal way individual-level attitudes towards those dimensions. 
		\end{itemize}

	\column{0.3\textwidth}
\begin{table}[h]
\begin{center}
{\renewcommand{\arraystretch}{2}%
\scalebox{0.3}{
\hspace{-2cm}
\begin{tabular}{cc}
\hline
\multicolumn{1}{|c|}{{\bf Candidate 1}}   & \multicolumn{1}{c|}{{\bf Candidate 2}}  \\ \hline
\multicolumn{1}{|c|}{\small{Media CAN confront the government}}    & \multicolumn{1}{c|}{\small{Media CANNOT confront the government}}   \\ \hline
\multicolumn{1}{|c|}{\small{President CANNOT rule without Congress}}    & \multicolumn{1}{c|}{\small{President CAN rule without Congress}}   \\ \hline
\multicolumn{1}{|c|}{\small{Citizens CANNOT vote in the next two elections}}    & \multicolumn{1}{c|}{\small{Citizens CANNOT vote in the next two elections}}   \\ \hline
\multicolumn{1}{|c|}{\small{Citizens CAN run for office for the next two elections}}    & \multicolumn{1}{c|}{\small{Citizens CAN run for office for the next two elections}}   \\ \hline
\multicolumn{1}{|c|}{\small{Citizens CAN associate with others and form groups}}    & \multicolumn{1}{c|}{\small{Citizens CANNOT associate with others and form groups}}   \\ \hline
\multicolumn{2}{c}{\texttt{Which of these candidates represents the lesser of the two evils for you?}} \\ \hline
\multicolumn{1}{|c|}{\texttt{Candidate 1} {\large$\square$}} & \multicolumn{1}{c|}{\texttt{Candidate 2} {\large$\square$}} \\ \hline
\end{tabular}}
}
\end{center}
\end{table}
\end{columns}
\end{frame}


\miniframesoff % 1 mapa
\begin{frame}\frametitle{{\scriptsize{``{\input{/Users/hectorbahamonde/research/Tobalaba/title.txt}\unskip}.''}}}	\label{tobalaba}	
	\begin{columns}\column{0.6\textwidth}
		\begin{itemize}
			\item {\bf Novel data}.
			\item {\color{ForestGreen}Identification strategy}: 
				\begin{enumerate}
					\item Aerodrome is located right in the middle of the wealthiest municipalities. % mapa
					\item Daily arrivals/departures of an aerodrome {\bf \emph{strictly} used by the elites}. % mapa
				\end{enumerate}
			%\item Working class citizens faced heavy control in buses and cars. % foto y series de tiempo
			%\item However, elites were able to use their airplanes mainly to go to their vacations houses.  % foto y series de tiempo 
			%\item {\bf Natural experiment} and {\bf RDD}: Intervention---confinement measures----are known and not assigned at random. % dos series de tiempo  

		\end{itemize}

	\column{0.4\textwidth}
		\begin{figure}[H]
		\vspace{-1cm}
		\hspace{-7mm}
		\includegraphics[scale=0.25]{/Users/hectorbahamonde/research/Tobalaba/map.pdf}
		\end{figure}
\end{columns}
\end{frame}








\subsection{Abstracts}

% todos los papers, incluidos los publicados






\end{document}

