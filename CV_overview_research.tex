%%%%%%%%%%%%%%%%%%%%
% Overview Research
%%%%%%%%%%%%%%%%%%%%
\subsection*{Research and Teaching Interests (undergraduate and graduate)}

Democratic backsliding, authoritarian values, populism, state formation, inequality, comparative political economy, political development, causal inference, experimental methods (natural, lab and survey-based) and statistical methods.

\subsection*{Dissertation}

	My dissertation argues that sectoral economic conflicts fostered state-building in Latin America. Using fine-grained historical case study comparisons, sectoral outputs from 1900 to the present, panel data and time-series econometric techniques, and a novel earthquake dataset (to measure state capacities), I find that industrial expansion altered the post-colonial political balance, thus putting heavy pressures for the implementation of tax institutions. In turn, fiscal expansion fostered both political development and economic growth.
