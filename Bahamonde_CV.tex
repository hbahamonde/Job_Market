% LaTeX Curriculum Vitae Template
%
% Copyright (C) 2004-2009 Jason Blevins <jrblevin@sdf.lonestar.org>
% http://jblevins.org/projects/cv-template/
%
% You may use use this document as a template to create your own CV
% and you may redistribute the source code freely. No attribution is
% required in any resulting documents. I do ask that you please leave
% this notice and the above URL in the source code if you choose to
% redistribute this file.

\documentclass[letterpaper]{article}

\usepackage{hyperref}
\usepackage{geometry}

% Comment the following lines to use the default Computer Modern font
% instead of the Palatino font provided by the mathpazo package.
% Remove the 'osf' bit if you don't like the old style figures.
\usepackage[T1]{fontenc}
\usepackage[sc,osf]{mathpazo}

% Set your name here
\def\name{Hector Bahamonde}

% Replace this with a link to your CV if you like, or set it empty
% (as in \def\footerlink{}) to remove the link in the footer:
\def\footerlink{}
% \href{http://www.hectorbahamonde.com}{www.HectorBahamonde.com}

% The following metadata will show up in the PDF properties
\hypersetup{
  colorlinks = true,
  urlcolor = blue,
  pdfauthor = {\name},
  pdfkeywords = {political science, economic development, methods},
  pdftitle = {\name: Curriculum Vitae},
  pdfsubject = {Curriculum Vitae},
  pdfpagemode = UseNone
}

\geometry{
  body={6.5in, 8.5in},
  left=1.0in,
  top=1.25in
}

% Customize page headers
\pagestyle{myheadings}
\markright{{\tiny \name}}
\thispagestyle{empty}

% Custom section fonts
\usepackage{sectsty}
\sectionfont{\rmfamily\mdseries\Large}
\subsectionfont{\rmfamily\mdseries\itshape\large}

% Other possible font commands include:
% \ttfamily for teletype,
% \sffamily for sans serif,
% \bfseries for bold,
% \scshape for small caps,
% \normalsize, \large, \Large, \LARGE sizes.

% Don't indent paragraphs.
\setlength\parindent{0em}

% Make lists without bullets
\renewenvironment{itemize}{
  \begin{list}{}{
    \setlength{\leftmargin}{1.5em}
  }
}{
  \end{list}
}

\begin{document}

% Place name at left
%{\huge \name}

% Alternatively, print name centered and bold:
\centerline{\huge \bf \name}

\vspace{0.25in}

\begin{minipage}{0.45\linewidth}
  Rutgers University, New Brunswick \\
  Political Science Department \\
  Hickman Hall \\
  New Brunswick, NJ 08901\\
  \\
  \\
\begin{footnotesize}
 CV last updated: \today. \\
 Download last CV \href{https://github.com/hbahamonde/Job_Market/raw/master/Bahamonde_CV.pdf}{\texttt{here}}.\\
 Download latest application materials \href{http://www.hectorbahamonde.com}{\texttt{here}}.
\end{footnotesize}

\end{minipage}
\begin{minipage}{0.45\linewidth}
  \begin{tabular}{ll}
    \texttt{cp}: & (732) 318-9650 \\
    \texttt{e}: & \href{mailto:hector.bahamonde@rutgers.edu}{Hector.Bahamonde@Rutgers.edu} \\
    \texttt{w}: & \href{http://www.hectorbahamonde.com}{www.HectorBahamonde.com}
    \\
    \\
    \\
    \\
    \\
    \\
  \end{tabular}
\end{minipage}


%\section*{Personal}
%
%\begin{itemize}
%\item Born on November 4th, 1984.
%\item Chilean Citizen.
%\end{itemize}


\section*{Education}

\begin{itemize}
  \item Ph.D. Political Science, Rutgers University (2017, expected).
  	\begin{itemize}
  		\item First field: {\small Comparative Politics}. \\ Second field: {\small Quantitative Methods}.
	\end{itemize}

  \item M.A. Political Science, Rutgers University, 2014.

  \item B.A. Political Science, Pontifical Catholic University of Chile, 2009.
\end{itemize}


%\section*{Employment}
%
%\begin{itemize}
%\item PLACE
%\end{itemize}


\section*{Research}

\emph{Research Interests: comparative political economy, economic development, Latin American politics, causal inference, experimental methods and quantitative methods}.

\subsection*{Dissertation}

My dissertation argues that the modernization of the fiscal apparatus was the product of an inter-elite bargaining process between the agricultural and industrial classes. The dissertation is embedded into a larger book project, where I analyze how these structural transformations helped states to make institutional investments that lead to the formation of states with higher capacities. Using fine-grained historical case study comparisons, sectoral outputs from 1900 to the present, time-series econometric techniques and a novel earthquake dataset (to measure state capacities), I find that industrial elites challenged the traditional/agricultural sector, creating the incentives to self-impose state institutions.



\subsection*{Peer-Reviewed Journal Articles}

\begin{itemize}
\item Pilar Giannini, Hector Bahamonde, Juan Pablo Luna, Rodolfo L\'opez, Mart\'in Ordo\~nez, Gonzalo Recart. 2011. \href{http://www.revistacienciapolitica.cl/rcp/wp-content/uploads/2013/09/07_vol_31_2.pdf}{\emph{El Secreto de mi \'Exito: Parte II. Los Caminos a Valpara\'iso en 2009}}. Revista de Ciencia Pol\'itica. 31(2), 285-310, 2010.
\end{itemize}



\subsection*{Working Papers}

\begin{itemize}
\item Hector Bahamonde. \href{https://github.com/hbahamonde/Clientelism_paper/raw/LARR_Revisions/Bahamonde_Clientelism_Paper.pdf}{\emph{Aiming Right at You: Group vs. Individual Clientelistic Targeting in Brazil}}.
\item Hector Bahamonde. \emph{Does the US have Healthier Democratic Values? Measuring Vote-Selling Elasticities Using a List and Conjoint Experiments}.
\item Hector Bahamonde. \emph{Structural Transformations and The Political Roots of Fiscal Capacities in Latin America}.
\item Hector Bahamonde. \emph{Inter-Sectoral Competition and State Capacities in Latin America: The Negative Link between the Industrial and the Agricultural Sectors}.
\item Hector Bahamonde. \emph{Measuring State Capacities in LA using an Infrastructural Approach: Earthquakes and Institutional Development, 1900-2000}.
\item Richard Lau (Rutgers), Mona Kleinberg (UMass - Lowell), Hector Bahamonde. \emph{Testing the Online Model of Candidate Evaluation in a More Realistic Environment}.
\end{itemize}

\subsection*{Other Publications}

\begin{itemize}
\item Juan Pablo Luna, Hector Bahamonde, Germ\'an Bidegain, Roody Reserve, Giancarlo Visconti. \emph{?`Estable Pero Sin Ra\'ices? Los Partidos Pol\'iticos Chilenos en la Opini\'on P\'ublica}, in \emph{Cultura Pol\'itica de la Democracia en Chile, LAPOP Country Report, 2010}.
\end{itemize}

\subsection*{Conferences}

\begin{itemize}
\item American Political Science Association (APSA): 2016.
\item Latin American Studies Association (LASA): 2014, 2015, 2016.
\item Western Political Science Association (WPSA): 2015, 2016.
\item Southern Political Science Association (SPSA): 2015.
\end{itemize}


\section*{Grants, Fellowships, \& Awards}

\begin{itemize}
\item Conference Travel Support (2013, 2014, 2015, 2016)
\item Teaching Assistantship Appointment (2015, 2016, 2017).
\item Small Grant Fund for Research on Latin America (Center for Latin American Studies, Rutgers).
\item Teaching Assistant Professional Development Fund (2015, 2016).
\item Pre-Dissertation Award (2016).
\item Experimental Workshop Award (2015).
\item Jerome M. Clubb Scholarship (2013, ICPSR).
\item Excellence Fellowship: full tuition plus health benefits (2012, Graduate School, Rutgers).
\end{itemize}


\section*{Teaching}

\begin{itemize}
\item ``Introduction to Quantitative Methods'' (TA, \emph{Graduate}, \texttt{fall}: 2015).
\item ``American Government'' (TA, \emph{Undergraduate}, \texttt{fall}: 2014 to \texttt{fall}: 2016).
\item ``Math Camp for Political Scientists'' (Instructor, \emph{Graduate}, \texttt{winter}: 2015).
\end{itemize}


\section*{Service}

\begin{itemize}
\item Comparative/International Political Economy Job Search (2015).
\item Discussant ``Emerging Trends Topics'' talks (2014, 2015).
\item Graduate Student mentor (2014, 2015, 2016).
\end{itemize}


\section*{Misc}

\subsection*{Languages}
English, Spanish (native).


\subsection*{Software}
\texttt{R}, \texttt{Stata}, \LaTeX.

\subsection*{Methods Training}
Time Series I (ICPSR), Introduction to Bayesian Analysis (ICPSR), Time Series II (ICPSR), Advanced MLE - Panel Data (ICPSR), Experimental Political Science (RU), Advanced Research in Political Economy (RU), Formal Modeling I (Princeton), Advanced Econometrics (RU, Economics), Introduction to Game Theory (ICPSR), Regression III: Advanced Methods (ICPSR), Measurement, Scaling and Dimensional Analysis (ICPSR), Causal Inference for the Social Sciences (ICPSR), Regression Analysis (RU, Statistics), Maximum Likelihood Estimation for GLMs (RU).




\bigskip



\end{document}
