%----------------------------------------------------------------------------------------
%	PACKAGES AND OTHER DOCUMENT CONFIGURATIONS
%----------------------------------------------------------------------------------------

\documentclass[10pt,stdletter,dateno,sigleft]{newlfm} % Extra options: 'sigleft' for a left-aligned signature, 'stdletternofrom' to remove the from address, 'letterpaper' for US letter paper - consult the newlfm class manual for more options

%\usepackage{charter} % Use the Charter font for the document text
\usepackage{graphics}

\usepackage{standalone}

%% with this patch, the newlfm should work (BEGIN)
\usepackage{etoolbox}
\makeatletter
\patchcmd{\@zfancyhead}{\fancy@reset}{\f@nch@reset}{}{}
\patchcmd{\@set@em@up}{\f@ncyolh}{\f@nch@olh}{}{}
\patchcmd{\@set@em@up}{\f@ncyolh}{\f@nch@olh}{}{}
\patchcmd{\@set@em@up}{\f@ncyorh}{\f@nch@orh}{}{}
\makeatother
%% with this patch, the newlfm should work (END)


\newsavebox{\Luiuc}\sbox{\Luiuc}{\parbox[b]{1.75in}{\vspace{0.5in}
\includegraphics[width=1.3\linewidth]{/Users/hectorbahamonde/Job_Market/logo2.jpg}}} % Company/institution logo at the top left of the page
\makeletterhead{Uiuc}{\Lheader{\usebox{\Luiuc}}}

\newlfmP{sigsize=50pt} % Slightly decrease the height of the signature field
\newlfmP{addrfromphone} % Print a phone number under the sender's address
\newlfmP{addrfromemail} % Print an email address under the sender's address
\PhrPhone{\texttt{p}} % Customize the "Telephone" text
\PhrEmail{\texttt{e}} % Customize the "E-mail" text

\lthUiuc % Print the company/institution logo


\usepackage{color} % for colors

\usepackage{hyperref}
\hypersetup{
    %bookmarks=true,         % show bookmarks bar?
    unicode=false,          % non-Latin characters in Acrobat’s bookmarks
    pdftoolbar=true,        % show Acrobat’s toolbar?
    pdfmenubar=true,        % show Acrobat’s menu?
    pdffitwindow=true,     % window fit to page when opened
    pdfstartview={FitH},    % fits the width of the page to the window
    pdftitle={My title},    % title
    pdfauthor={Author},     % author
    pdfsubject={Subject},   % subject of the document
    pdfcreator={Creator},   % creator of the document
    pdfproducer={Producer}, % producer of the document
    pdfkeywords={keyword1} {key2} {key3}, % list of keywords
    pdfnewwindow=true,      % links in new window
    colorlinks=true,       % false: boxed links; true: colored links
    linkcolor=blue,          % color of internal links (change box color with linkbordercolor)
    citecolor=blue,        % color of links to bibliography
    filecolor=blue,      % color of file links
    urlcolor=blue           % color of external links
}

\hypersetup{
  colorlinks = true,
  urlcolor = blue,
  pdfpagemode = UseNone
}


%%% bib begin
\usepackage[american]{babel}
\usepackage{csquotes}
%\usepackage[style=chicago-authordate,doi=false,isbn=false,url=false,eprint=false]{biblatex}

\usepackage[authordate,isbn=false,doi=false,url=false,eprint=false]{biblatex-chicago}
\DeclareFieldFormat[article]{title}{\mkbibquote{#1}} % make article titles in quotes
\DeclareFieldFormat[thesis]{title}{\mkbibemph{#1}} % make theses italics

\AtEveryBibitem{\clearfield{month}}
\AtEveryCitekey{\clearfield{month}}

\addbibresource{/Users/hectorbahamonde/Bibliografia_PoliSci/library.bib} 
\addbibresource{/Users/hectorbahamonde/Bibliografia_PoliSci/Bahamonde_BibTex2013.bib} 

% USAGES
%% use \textcite to cite normal
%% \parencite to cite in parentheses
%% \footcite to cite in footnote
%% the default can be modified in autocite=FOO, footnote, for ex. 
%%% bib end

%----------------------------------------------------------------------------------------
%	YOUR NAME AND CONTACT INFORMATION
%----------------------------------------------------------------------------------------

%\namefrom{\vspace{-3cm}Hector Bahamonde, PhD} % Name

\addrfrom{
{\vspace{-2cm}}\\ % Date
[12pt]
{\bf Hector Bahamonde, PhD}\\
{\color{blue}{\bf Senior Researcher}}\\
\emph{\href{https://invest.utu.fi/welfare-research-and-ecosystem/investhub/}{INVESThub}, \href{https://invest.utu.fi}{INVEST Research Flagship Centre}}\\
{\color{blue}University of Turku}\\
Turku, Finland
}

\phonefrom{+358 (0) 40-362-1438} % Phone number

\emailfrom{\href{mailto:hector.bahamonde@utu.fi}{hector.bahamonde@utu.fi} \\ 
\texttt{w}: \href{http://www.hectorbahamonde.com}{www.HectorBahamonde.com}\\
\today}% Link to last version


%----------------------------------------------------------------------------------------
%	ADDRESSEE AND GREETING/CLOSING
%----------------------------------------------------------------------------------------

%\greetto{\vspace{-1cm}Dear Members of the Search Committee,} % Greeting text
%\closeline{\vspace{-1cm}\includegraphics[width=3cm]{/Users/hectorbahamonde/Administracion/signature.pdf} \hspace{-4cm}} % Closing text




%----------------------------------------------------------------------------------------

\begin{document}
\begin{newlfm}

%----------------------------------------------------------------------------------------
%	LETTER CONTENT
%----------------------------------------------------------------------------------------

%\vspace{-1cm}
\vspace{-2cm}I am writing to express my interest in the Senior Researcher position at InvestHub, situated within the Faculty of Social Sciences and the INVEST Flagship and Research Centre at the University of Turku. Presently, I hold the same role as a Senior Researcher at this institution. In this letter, I will describe my research and teaching interests, highlight my accomplishments during my two years here at InvestHub, and importantly, show how these relate to the activities of the research group.

% my research interests
{\bf My research interests} are closely aligned with the core focus areas of InvestHub. My work primarily revolves around the fields of political economy, inequality, and the application of experimental and statistical methods. Specifically, \emph{my research delves into the political consequences of economic inequality, utilizing experimental and quantitative methodologies}. This includes quasi-experimental designs \parencite{Bahamonde2018}, lab-based experimental settings \parencite{Bahamonde2022b}, natural experimental settings \parencite{Bahamonde:2023}, and survey experimental settings \parencite{Bahamonde2020a}. Moreover, my expertise extends to econometric methods. This is evident in our collaborative work, which demonstrates a significant association between inequality and the evolution of democratic systems \parencite{Bahamonde2021}. This research not only contributes to the academic discourse but also provides practical insights into the dynamics of political and economic structures.

Since my appointment as a Senior Researcher at InvestHub in August 2021, I have successfully published three papers in top-tier journals under the University of Turku's INVEST affiliation. Additionally, another significant publication$^{\diamond}$ was released during my transition period to InvestHub: 

\begin{enumerate}
  \item \href{https://doi.org/10.1111/pops.12940}{Hector Bahamonde and Outi Sarpila. 2023. ``Physical Appearance and Elections: An Inequality Perspective.'' \emph{Political Psychology}.}
  \item \href{https://doi.org/10.1016/j.electstud.2022.102497}{Hector Bahamonde and Andrea Canales. 2022. ``Electoral Risk and Vote Buying, Introducing Prospect Theory to the Experimental Study of Clientelism.'' \emph{Electoral Studies}.}
  \item \href{https://doi.org/10.1016/j.ejpoleco.2021.102048}{Hector Bahamonde and Mart Trasberg. 2021. ``Inclusive Institutions, Unequal Outcomes: Democracy, State Capacity and Income Inequality.'' \emph{European Journal of Political Economy}, 1---36.}
  \item[$\diamond$] \href{https://link.springer.com/article/10.1057/s41269-020-00174-4}{Hector Bahamonde. 2022. ``Still for Sale: The Micro-Dynamics of Vote Selling in the United States, Evidence from a List Experiment.'' \emph{Acta Politica} 57 (1): 73---95.}
\end{enumerate}

This consistent output of high-quality research underscores my commitment to contributing valuable knowledge in my field and reflects the synergistic relationship between my research endeavors and InvestHub's objectives.

% my teaching interests
{\bf My teaching interests} and experiences align with the objectives and ethos of InvestHub. Since I began my position as a Senior Researcher at InvestHub, I have had the privilege of teaching two advanced courses: ``\href{https://github.com/hbahamonde/Exp_Soc_Science/raw/main/Bahamonde_Exp_Soc_Sci.pdf}{Experimental Social Science}'' and ``\href{https://github.com/hbahamonde/OLS/raw/master/Bahamonde_OLS.pdf}{Quantitative Methods: Ordinary Least Squares in \texttt{R}}.'' These courses, designed for Master's level students, have allowed me to delve deep into these subjects, fostering a rich learning environment. This Fall 2023, I am taking on the role of thesis supervisor for a Master's student, a responsibility I approach with great enthusiasm and dedication. Looking ahead to 2024, I am excited to expand my supervisory role to include PhD students. This progression in my teaching journey not only reflects my commitment to academic excellence but also aligns seamlessly with the center's mission to nurture and develop advanced research skills in students.

% Institutional building INVESThub
{\bf Institutional building at INVESThub}. Upon assuming my role as a Senior Researcher at INVESThub, I was honored to be appointed as its first full-time Research Director. This role placed me at the helm of coordinating both scientific and administrative activities. Given that INVESThub was relatively new at the time, my primary objectives were to strengthen its institutional capacities, broaden its influence, and establish a compelling scientific narrative. When I took over the center's direction, it was akin to starting from scratch. My initial endeavor was to cultivate a community of researchers from various social science disciplines, unified by a shared interest in interventions, experimental and quasi-experimental methodologies. To facilitate this, \href{https://invest.utu.fi/news/investhub-brings-together-social-scientists-interested-in-experimental-research/}{I myself initiated a series of monthly talks named ``Brown Bags.''} These sessions provided a platform for researchers from the University of Turku and beyond to present their work, particularly focusing on experimental methodologies. The Brown Bags have since become a successful forum for both scientific and social exchange, with me continuing as the sole organizer. 

Another significant development occurred when researchers from Jyväskylä University (another Finnish university) expressed interest in participating in the Brown Bags, despite not having papers or data ready. Seizing this opportunity for community building, I established a separate series of talks named ``InvestHub Workshops.'' These workshops allowed researchers to present their designs and discuss hypotheses, internal validity, ethical considerations, and methodological challenges \emph{before} conducting their experiments. The idea was to help researchers during the \emph{initial} stages of their experimental field works. This format encouraged a collaborative environment where researchers could bring ``more questions than answers.'' {\bf Both the ``InvestHub Brown Bags'' and the ``InvestHub Workshops,'' which I created and have solely organized, remain central to the center's activities, even after my role as InvestHub's Research Director ended. These initiatives have become integral to INVESThub's identity}.

In addition to these community-building efforts, I integrated practical training into the curriculum. I ensured that students enrolled in our ``\href{https://github.com/hbahamonde/Exp_Soc_Science/raw/main/Bahamonde_Exp_Soc_Sci.pdf}{Experimental Social Science}'' course participated in both the Brown Bags and Workshops, reinforcing my belief in the importance of hands-on learning as the next step for INVESThub. During my time as Research Director, I also oversaw several recruitment processes, including hiring an IT expert. This involved conducting meetings to accurately determine our IT requirements, thereby streamlining the job profiling and advertisement process. Lastly, I organized the inaugural ``\href{https://invest.utu.fi/wp-content/uploads/2022/10/Scientific-Seminar-VII-Programme.pdf}{InvestHub Panel}'' at the INVEST Scientific Seminar, an internal conference.\footnote{``Parallel Session 2.''} This event provided a platform for our researchers and affiliates to present their papers and engage in stimulating discussions, further enhancing the academic environment at INVESThub.

{\bf Collaborative work within Invest and beyond}. My time at the University of Turku has been marked by highly productive collaborative endeavors. I am a strong proponent of multidisciplinary collaboration, and my work at the university has been a testament to this belief. A prime example of such collaboration is my partnership with \href{https://www.utu.fi/en/people/outi-sarpila Outi Sarpila}{Professor Outi Sarpila}, a professor of Sociology at the University of Turku. Together, we have developed and refined a groundbreaking framework. Our joint research, particularly highlighted in \textcite{Bahamonde:2023}, reveals how voters' perceptions are influenced by the socioeconomic implications of candidates' physical appearances. In this natural experiment, we discovered that Finnish citizens tend to favor candidates who appear to have upper-class occupations. This bias is especially pronounced in the case of female candidates, who face significant electoral disadvantages when perceived as working class. Our findings, published in \href{https://www.scimagojr.com/journalrank.php?category=3320}{\emph{Political Psychology}}, underscore the intricate interplay between physical appearance and electoral success. My collaborative efforts extend within InvestHub as well. I am currently working with \href{https://www.utu.fi/en/people/aki-koivula}{Aki Koivula}, a fellow Senior Researcher and political sociologist at InvestHub. Our joint research focuses on a novel interpretation of how economic inequality influences the rise of populist parties. We are in the process of drafting a working paper on this topic, which will be presented at several international conferences next year. I also firmly believe in the importance of extending collaboration beyond the University of Turku. In line with this, my economist co-author from Chile and myself published a piece in \href{https://doi.org/10.1016/j.electstud.2022.102497}{\emph{Electoral Studies}}. Also I  am currently engaged in a series of projects with colleagues from \AA bo Akademi, another prestigious Finnish university based in Turku. My role in these projects is pivotal, as I am responsible for developing and overseeing a number of survey experiments. This collaboration not only broadens the scope of my research but also enriches the academic dialogue between institutions, fostering a more diverse and comprehensive understanding of our shared research interests.

\newpage

% outreach
{\bf Active collaboration and contribution to my own discipline}. My commitment to advancing my field extends beyond individual research; I have actively engaged in collaborative efforts aimed at strengthening the discipline as a whole. A testament to this is my role in co-organizing a comprehensive section of panels, also referred to as ``workshops,'' for the 2022 \emph{Finnish Political Science Association} (FPSA) conference. Building on this successful experience, I am currently preparing to undertake a similar responsibility for the upcoming 2024 FPSA conference, which will be hosted at the University of Turku. This role not only underscores my dedication to fostering my discipline but also highlights my skills in bringing together experts and facilitating meaningful discussions in political science. In addition to these organizational endeavors, I have been a speaker at both national and international conferences. My engagement has spanned various regions, including Europe, the United States, and South America, where I have delivered a total of 12 talks during these two years at InvestHub. This extensive speaking experience not only reflects my expertise and recognition in the field but also demonstrates my commitment to disseminating knowledge and contributing to the field. Overall, my efforts in organizing significant conference sections and delivering numerous talks are integral to my professional identity. Finally, I am an \emph{Editorial Board member} of the \href{https://www.springer.com/journal/43545}{Social Sciences} journal (Springer).

% Grants and future research
{\bf Securing grants for my research plan.} I am currently in the process of applying for funding to support my long-term research project, which is focused on a critical examination of the rise of populist, far-right, and far-left parties in contemporary democracies. This project is rooted in the hypothesis that escalating economic inequality is sharpening perceived losses among less privileged groups, thereby fueling increased support for these political factions. While previous studies have explored the impact of inequality on radical political preferences, my research brings a unique perspective by emphasizing the concept of ``loss aversion.'' This principle, which suggests that people are more sensitive to losses than to equivalent gains, is a cornerstone of my analysis, as extensively discussed in \textcite{Bahamonde2022b} and \textcite{Bahamonde2023}. It posits that voters' decisions are often influenced by perceived losses in times of economic and political uncertainty. Methodologically, my project challenges the conventional reliance on survey research in this field. Acknowledging the potential biases inherent in surveys, I am incorporating a diverse array of research methods. This includes panel laboratory experiments, survey experiments, register data, and advanced econometric techniques, providing a more comprehensive and nuanced understanding of the subject.

To secure the necessary funding for this ambitious project, I have submitted applications to several prestigious funding bodies. These include the \emph{Kone Foundation} (project \#202303858),\footnote{Kone Foundation ``is an independent non-profit organisation that awards grants to promote academic research, culture and the arts'' in Finland (https://koneensaatio.fi/en/).} the \emph{Finnish Cultural Foundation} (project \#575159CE),\footnote{``The Finnish Cultural Foundation is a private foundation dedicated to promoting art and science'' (https://skr.fi/en).} and the \emph{European Research Council}'s ``Starting Grant'' (project \#101163663). Additionally, I plan to submit an application to The Research Council of Finland in January 2024. This strategic approach to funding underscores my recognition of the importance of external resources in academic research. Overall, the list of institutions to which I have applied reflects my commitment to seeking support from the most competitive and prestigious funding bodies in Finland and Europe. The substantive long-term goals and methodologies of my research are further detailed in my Research Plan addendum. This comprehensive approach to securing funding demonstrates my dedication to advancing this significant area of study and contributing valuable insights to the understanding of political dynamics in modern democracies.

% personal
{\bf A personal note.} Finally, living in Finland, and particularly in the city of Turku, has been an exceptionally positive experience for my family and me. This beautiful country has become a cherished home where we are raising our three young children, aged one, five, and seven. We have all fallen deeply in love with the unique aspects of Finland. These personal and familial experiences have significantly contributed to my motivation to reapply for this position. The integration into Finnish life has been a nice journey of cultural immersion for my family. My wife, embracing our new home, has taken the initiative to learn the language by enrolling in a `Finnish for Beginners' course. This step towards cultural integration reflects our family's commitment to embracing and respecting our host country's traditions and way of life. Our children, Tobias and Olimpia, have been fully immersed in the Finnish education system since October 2021. Tobias attends an all-Finnish language school, while Olimpia is in a Finnish daycare. And one year ago we had our third child at Turku Hospital (TYKS)---he soon will go to the same daycare. This exposure to the local language and culture from a young age is invaluable, ensuring that they grow up with a deep understanding and appreciation of their surroundings. Furthermore, the immigration process has been smooth for us, especially since my wife holds a German passport. This EU citizenship has eased our transition and is particularly beneficial now as my wife seeks employment opportunities in Turku. As an English Literature and Linguistics major, her job search is not just about professional fulfillment but also about contributing to the community that has welcomed us so warmly. In summary, the personal and family satisfaction we have found in Finland greatly reinforces my desire to continue my professional journey here.

Thank you for considering my application. I look forward to hearing from you. Should you have any questions about my application, please do contact me. I'll be happy to clarify any doubts you might have.




\vspace{3cm}

{\hspace{12cm}Hector Bahamonde, PhD}


\newpage

\printbibliography



%----------------------------------------------------------------------------------------

\end{newlfm}
\end{document}
