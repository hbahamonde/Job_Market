\vspace{-2cm}{\bf \huge Teaching Statement}\\

{\bf Teaching Philosophy.} As an instructor, my goal has always been to sow the seed of curiosity, because it is the first stepping stone of learning. One of the major challenges of teaching comparative politics is that it is a stream of conflicting theories, approaches, and methodologies. My belief is that this might be overwhelming for students. Hence, my teaching philosophy is to serve as a \emph{guide} in the process of discovering what comparative politics, democracy, development, and political economy are.

% UOH
At O'Higgins University I teach seminar courses in \href{https://github.com/hbahamonde/Ciencia_Politica_I/raw/master/Bahamonde_Ciencia_Politica_I.pdf}{comparative politics} and \href{https://github.com/hbahamonde/Ciencia_Politica_II/raw/master/Bahamonde_Ciencia_Politica_II.pdf}{comparative political economy}, as well as the methods sequence in the Economics program (both \href{https://github.com/hbahamonde/OLS/raw/master/Bahamonde_OLS.pdf}{OLS} and \href{https://github.com/hbahamonde/MLE/raw/master/Bahamonde_MLE.pdf}{MLE-causal inference}). After a good number of years teaching in the United States, I've decided to make use of my experience here at home. What I've discovered, is that no matter what the country is, the needs are the same: students need \emph{proactive mentors} in their seek of knowledge. 

% tulane
In the spring of 2018, while I was at {\bf Tulane University}, I taught \href{https://github.com/hbahamonde/Comparative_Politics_UGRAD/raw/master/Bahamonde_Comparative_Politics_Syllabus_UGRAD.pdf}{Introduction to Comparative Politics}. It was a really enjoyable experience, for me and my students (check my \href{https://github.com/hbahamonde/Job_Market/raw/master/Bahamonde_Research_Portafolio.pdf}{teaching evaluations}). I designed the syllabus not only thinking about how to retain the interest of political science / global studies majors, but also about how to captivate and motivate prospective students. And while I put heavy weight on participation, my experience teaching at Rutgers taught me how to create a classroom environment of intellectual curiosity and mutual respect. My number one rule is to approach all these big questions by presenting the material in such way that my students feel intrigued about it. I believe this to be the main ingredient to train individuals who can think critically and navigate the major debates in the field---not only from a theoretical perspective, but also from an applied point of view.

% TA @ RU
As a teaching assistant at {\bf Rutgers University}, I was fortunate enough to teach in one of the most diverse schools in the country. As an engaging instructor, I took pedagogical advantage of this situation by bringing into the classroom many examples from different parts of the world. Teaching in such a diverse environment gave me extensive training in how to approach controversial issues, and also in how to present the material in an interesting way for \emph{all} students, regardless of their different cultural and economic backgrounds. You can access my diversity statement \href{http://github.com/hbahamonde/Job_Market/raw/master/Bahamonde_Diversity_Statement.pdf}{here}.


% TA @ RU: Methods course.
I have not only taught at the undergraduate level, but I have also served as a {\bf teaching assistant at the graduate level}. In the fall of 2015, I served as the TA of the \emph{Introduction to Statistics} course taught by Professor Beth Leech. It was a great experience. For instance, I gave a talk on how to present statistical models in an appealing and intuitive way. I engaged my fellow graduate students in a way such that they could not only \emph{see} how statistical results should look like, but also how to actually do it. 

% Teaching Math Camp
In the winter of 2015, I had the opportunity to {\bf teach the} \href{https://github.com/hbahamonde/Math-Camp/raw/master/Syllabus/Math_Camp_Syllabus.pdf}{{\bf Math Camp and Introduction to Computing}} course that ran all day, for an entire week. The course was intended for first-year graduate students, and it covered all necessary elements to perform well in the methods sequence. In general, this is a complex subject matter to teach; it requires superb organizational and teaching skills. I decided then to adopt a ``\emph{no child left behind}'' policy. This is very important to me, not only in this particular context, but in any class I have teach. Shy students with unanswered questions perceive no benefit if the instructor is \emph{only} ``engaging.'' {\bf I believe it is fundamental to create an atmosphere of constructive learning, and an environment of tolerance that fosters the notion that \emph{we} (i.e, students and myself) are finding the possible answers \emph{together}}. That is why I feel it is fundamental to reward all sorts of possible questions. It is by asking multiple questions that we can stimulate an environment that cradles learning and curiosity. Almost every lecture I have ever given adapts to the students' questions, creating and environment of discussion and ``nutritive'' debate. {\bf Rephrasing and re-framing students' questions allows me to accomplish these goals while still sticking to the syllabus}. 

% Mentoring
An important aspect of belonging to an active academic community is the opportunity to {\bf mentor} students, both graduate and undergraduate. For this reason, I always served as a graduate student mentor. In doing so, I had the opportunity to help incoming graduate students with their transition to grad school. At the undergraduate level, I always provided advice to interested undergraduate students wanting to pursue a career and/or a PhD/MA in Political Science. \emph{As an undergraduate, I still remember how important mentoring for me was in my final decision to apply for graduate school}.

{\bf Teaching Interests.} Going forward, I would like to teach courses in comparative politics, political economy of development, democracy, state formation, Latin American politics, and applied methods courses (statistical and experimental). However, I can be quite flexible and take care of the demands the department has. Please check my \href{http://www.hectorbahamonde.com/research/}{\texttt{research agenda}} and see how my {\bf teaching and research interests} match. Below I describe a potential list of courses:
