\vspace{-2cm}{\bf \huge Teaching Statement}\\

{\bf Teaching Philosophy.} As an instructor, my goal has always been to sow the seed of curiosity, since that it is the first stepping stone of learning. One of the major challenges of teaching comparative politics is that, as any social science, it is a stream of conflicting theories, approaches, and methodologies. My belief is that this is overwhelming for students, hence my teaching philosophy is to serve as a \emph{guide} in the process of discovering what comparative politics is.


% tulane
In the spring semester of 2018, I am scheduled to teach the \emph{Introduction to Comparative Politics} course at Tulane University, for which I have developed a broad, but very interesting \href{https://github.com/hbahamonde/Comparative_Politics_UGRAD/raw/master/Bahamonde_Comparative_Politics_Syllabus_UGRAD.pdf}{syllabus}. I designed it having in mind not only thinking about how to retain the interest of political science / global studies mayors, but also thinking about how to captivate and motivate prospective students. And while I put heavy weight on participation, my experience teaching at Rutgers has taught me how to create a classroom environment of intellectual curiosity and collective respect. My number one rule is to approach all these big and interesting questions by presenting the material in such way that my students feel intrigued and curious about it. I believe this to be the main ingredient to form individuals who can think critically and navigate the major debates in the field. Not only from a theoretical perspective, but also from an applied point of view.

% TA @ RU
As a teaching assistant at Rutgers, I was fortunate enough to teach in one of the most diverse schools in the country. As an engaging instructor, I always took pedagogical advantage of this situation by bringing into the classroom many examples from different parts of the world. Given the diversity of the Rutgers' student body, it was almost always the case that I had a student from my example country. The opportunity I had of teaching in diverse environments gave me extensive training in how to approach controversial issues, and also in how to present the material in an interesting way for \emph{all} students, regardless of their different cultural and economic backgrounds. You can access my diversity statement \href{http://github.com/hbahamonde/Job_Market/raw/master/Bahamonde_Diversity_Statement.pdf}{here}.


% TA @ RU: Methods course.
I have not only taught at the undergraduate level, but I have also served as a teaching assistant at the graduate level too. In the fall of 2015, I served as the TA of the \emph{Introduction to Statistics} course taught by Professor Beth Leech. It was a great experience. For instance, I gave a talk on how to present statistical models in an appealing and intuitive way. I engaged my fellow graduate students in a way such that they could not only \emph{see} how statistical results should look like, but also how to actually do it. 

% Teaching Math Camp
In the winter of 2015, I had the opportunity to teach the \emph{Math Camp and Introduction to Computing} course that ran all day for an entire week. The course was intended for first-year graduate students, and it covered all necessary elements to perform well in the methods sequence. I designed the \href{https://github.com/hbahamonde/Math-Camp/raw/master/Syllabus/Math_Camp_Syllabus.pdf}{syllabus} so we could spend two days working on calculus, two days on matrix algebra, and one full day on computing. In general, this is a complex subject matter to teach; it requires superb organizational and teaching skills. I decided then to adopt a \emph{no child left behind} policy. This is actually very important to me. Not only in this particular context, but in any class I have taught. Shy students with unanswered questions perceive no benefit if the instructor is \emph{only} ``engaging.'' {\bf I believe it is fundamental to create an atmosphere of constructive learning, and an environment of tolerance that fosters the notion that \emph{we} (i.e, students and myself) are finding the possible answers \emph{together}}. That is why I feel it is fundamental to reward all sorts of possible questions. It is by asking multiple questions that we learn and stimulate an environment that cradles learning and curiosity. Almost every lecture I have ever given adapts to the students' questions, creating and environment of discussion and ``nutritive'' debate. Rephrasing and re-framing students' questions allows me to accomplish these goals while still sticking to the syllabus. 

% Mentoring
Finally, one important aspect of belonging to an active academic community is the opportunity to {\bf mentor} students, both graduate and undergraduate. For this reason, I always served as a graduate student mentor. In doing so, I had the opportunity to help incoming students with their transition into graduate school. At the undergraduate level, I always provided advice to interested undergraduate students wanting to pursue a career and/or a PhD/MA in political science. As an undergraduate, I still remember how important mentoring for me was in my final decision to apply for graduate school.

{\bf Teaching Interests.} Going forward, I would like to teach courses in comparative politics, political economy of development, Latin American politics, and applied methods courses. However, I can be quite flexible and take care of the demands the department has. Please check my \href{http://www.hectorbahamonde.com/research/}{\texttt{research agenda}} and see how my {\bf teaching and research interests} match. Below I describe a potential list of courses:
