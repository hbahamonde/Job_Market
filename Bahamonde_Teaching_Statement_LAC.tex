%----------------------------------------------------------------------------------------
%	PACKAGES AND OTHER DOCUMENT CONFIGURATIONS
%----------------------------------------------------------------------------------------

\documentclass[11pt]{letter} % Default font size of the document, change to 10pt to fit more text

%\usepackage{newcent} % Default font is the New Century Schoolbook PostScript font 
%\usepackage{helvet} % Uncomment this (while commenting the above line) to use the Helvetica font

% packages
\usepackage{hyperref} % for links
\usepackage[bottom]{footmisc} % footnotes at the bottom.
\usepackage{color} % for colors

\hypersetup{
  colorlinks = true,
  urlcolor = blue,
  pdfpagemode = UseNone
}

% Margins
\topmargin=-1in % Moves the top of the document 1 inch above the default
\textheight=9in % Total height of the text on the page before text goes on to the next page, this can be increased in a longer letter
\oddsidemargin=-10pt % Position of the left margin, can be negative or positive if you want more or less room
\textwidth=6.5in % Total width of the text, increase this if the left margin was decreased and vice-versa

%\let\raggedleft\raggedright % Pushes the date (at the top) to the left, comment this line to have the date on the right
\date{}


\begin{document}

%----------------------------------------------------------------------------------------
%	ADDRESSEE SECTION
%----------------------------------------------------------------------------------------

\begin{letter}{} 

%----------------------------------------------------------------------------------------
%	YOUR NAME & ADDRESS SECTION
%----------------------------------------------------------------------------------------

\begin{center}
\large\bf Hector Bahamonde \\ % Your name
Rutgers University\\
\vspace{20pt} \hrule height 1pt % If you would like a horizontal line separating the name from the address, uncomment the line to the left of this text
89 George St. \\ New Brunswick, NJ 08901 \\ (732) 318-9650 \\ 
{\normalfont\normalsize\href{mailto:hector.bahamonde@rutgers.edu}{Hector.Bahamonde@Rutgers.edu}} \\
{\normalfont\normalsize\href{http://www.hectorbahamonde.com}{www.HectorBahamonde.com}}\\
{\normalfont \scriptsize{
% Date
\vspace{5mm}\today\\
Download last version \href{http://github.com/hbahamonde/Job_Market/raw/master/Bahamonde_Teaching_Statement_LAC.pdf}{\texttt{{\color{red}here}}}}} % Link to last version


{\huge\vspace{8mm} Teaching Statement}
\end{center} 

\signature{\vspace{1cm}{\color{white}h.b.}} % Your name for the signature at the bottom

%----------------------------------------------------------------------------------------
%	LETTER CONTENT SECTION
%----------------------------------------------------------------------------------------
\opening{} 
 
{\bf Teaching Philosophy.} As a teacher, my goal has always been to sow the seed of curiosity, since that it is the first stepping stone of learning. One of the major challenges of teaching comparative politics is that, as any social science, it is a stream of conflicting theories, approaches and methodologies. My belief is that this could be overwhelming for students, hence my teaching philosophy is to serve as a \emph{guide} in the process of discovering comparative politics.

% undergrad
I did not start my teaching career as a graduate student, but rather, as an advanced undergraduate student in the \emph{Institute of Political Science} at the \emph{Catholic University of Chile}. As a teaching assistant, I would direct weekly discussion sessions of the entire sequence on history of political thought. It was during those years that I discovered many critical goals for my future. First, I wanted to continue on the path that I was following, which was teaching and researching (since I was also a RA). Second, I wanted to establish my teaching framework of approaching big questions by presenting the material in such a way that others felt intrigued and curious about it. I believe these are two of the main ingredients to form individuals who can think critically and navigate the major debates in comparative politics. 

% TA @ RU
As a teaching assistant at Rutgers, I have been fortunate enough to teach in one of the most diverse schools in the country. As an engaging teacher, I always take pedagogical advantage of this situation by bringing into the class many examples from different parts of the world. Given the diversity of Rutgers' student body, it is almost always the case that I have a student from my example country. Teaching in diverse environments has given me extensive training on how to approach controversial issues and how to present the material in a way that it is interesting for all students regardless of their different cultural backgrounds. 

% TA for Am. Govt.
Currently I am the head teaching assistant of the \emph{American Government} course, taught by Professor Ross Baker. In this course, we teach major developments in American politics, such as the origins of federalism, major trends in public opinion, the incentives political parties have during election times and other broad topics. Besides cultural diversity, this particular class always has students majoring in various different disciplines. These environments are twice as challenging given the lack of knowledge or even the lack of interest in the material presented. However, I like to present dry topics in such a way that the students feel directly ``affected'' by them. For example, in one lecture, we were analyzing the costs of political participation. Rather than address the problem from a dry or theoretical perspective, I chose to compare it with social media, an area that is relevant and familiar to the student population. In this manner, it was easier for them to think about what were the costs and benefits of each form of participation. 

% TA @ RU: Methods course.
I have not only taught at the undergraduate level, but I have also served as a teaching assistant at the graduate level. In the Fall of 2015, I served as the TA of the \emph{Introduction to Statistics} course that Professor Beth Leech taught. There, I gave a talk on how to present statistical models in an appealing and intuitive way. I engaged my colleagues in a way such that they could not only see how statistical results should look like but also how to actually produce them using a statistical package. 

% Teaching Math Camp
At Rutgers, teaching assistants are not allowed to teach semester-length courses. However, in the winter of 2015, I had the incredible opportunity to teach the \emph{Math Camp and Introduction to Computing} course that ran all day for an entire week. This course is intended for first-year graduate students and it covers all necessary elements to perform well in the methods sequence courses. I designed the \href{https://github.com/hbahamonde/Math-Camp/raw/master/Syllabus/Math_Camp_Syllabus.pdf}{syllabus} to spend two days working on calculus, two days on matrix algebra and one full day on computing. In general, this is a complex subject matter to teach and to learn; it requires superb organizational and teaching skills. I decided then to adopt a \emph{no child left behind} policy. This is actually very important to me as a teacher, not only in this particular context, but in any class I have participated in as a teacher or TA. Shy students with unanswered questions perceive no benefit if the instructor is \emph{only} ``engaging.'' I believe it is fundamental to create an atmosphere of constructive learning and an environment of tolerance that fosters the notion that \emph{we} (i.e, students and the instructor altogether) are finding the possible answers \emph{together}. That is why I feel it is fundamental to reward all sorts of possible questions. It is by asking multiple questions that we learn and stimulate an enviornment that cradles learning and curiosity. Almost every lecture I have ever given tunes into students' questions and in turn, makes them a topic of discussion and debate. Rephrasing and re-framing students' questions allows me to accomplish these goals while sticking to the syllabus. 

% Mentoring
Finally, one important aspect of being part of an active academic community is the need for mentoring and the possibility to give it, at both the graduate and undergraduate levels. For this reason, I serve as a graduate student mentor every year. In doing so, I have the opportunity to help incoming students with their transition into graduate school. Issues range from time management to computational working flows, to more substantive topics. At the undergraduate level, I always provide advice to interested undergraduate students wanting to pursue a career and/or a PhD/MA in political science. As an undergraduate, I still remember how important mentoring was for me in my final decision to apply for graduate school.

{\bf Teaching Interests.} Going forward, I would like to teach courses in comparative politics, political economy of development, Latin American politics and applied methods courses. However, within this range, I can be quite flexible. Below I describe a potential list of courses:

\begin{itemize}
\item Substantive Courses:
	\begin{itemize}
	\item Introduction to Comparative Politics (\href{https://github.com/hbahamonde/Comparative_Politics_UGRAD/raw/master/Bahamonde_Comparative_Politics_Syllabus_UGRAD.pdf}{syllabus}).
	\item Political Regimes and Regime Change.
	\item Political Economy of Development (\href{https://github.com/hbahamonde/Political-Economy-Intro-UGrad/raw/master/Pol_Econ_Dev_Syllabus_UGRAD.pdf}{syllabus}).
	\item Economic History and Political Economy.
	\item Introduction to Latin American Politics (\href{https://github.com/hbahamonde/Latin_American_Politics_UGRAD/raw/master/Bahamonde_Latin_American_Politics_Syllabus_UGRAD.pdf}{syllabus}).
	\end{itemize}
\item Methods courses:
	\begin{itemize}
	\item Applied Quantitative Methods in Political Science.
	\item Research Design / Epistemology in Political Science.
	\item Introduction to Quantitative Methods in Political Science.
	\item Experimental Methodology.
	\end{itemize}
\end{itemize}


{\bf Sample Student Evaluations}

{\scriptsize
\begin{itemize}

\item \emph{``The TA is very responsive when spoken to and is quick to answer questions via email. The TA's willingness to learn with us is also helpful in learning the material and allows us to have nice discussions in class.''}

\item \emph{``My TA showed he knew his subject material because he was able to answer hard and complicated questions efficiently despite it being obvious that English was not his first language.''}

\item \emph{``Hector showed me how to make connections with government terms. He made the big picture seem simpler for me.''}

\item \emph{``Over the break, I came to the conclusion that I want to major in political science. American Government was the first course I ever took related to political science, and I loved it.''}

\item \emph{``I am very grateful for your help and will definitely reach out to you to ask questions about Comparative Politics if that's what I eventually plan on doing. I feel like I'm very new to this whole field of study - mainly because I haven't been in the US for a very long time, and because of the way the government works so differently here than in Pakistan, where I'm from.''}

\item \emph{``The teaching assistant really helped me to think about all the ``why'' aspects of the material. Like for example, ``Why is this important?'' or ``Why does this relate to the material?''.''}

\item \emph{``The best TA in teaching the course material. Each recitation session is well compact with main concepts crucial for understanding the course material.''}

\item \emph{``As an international student who takes the course for requirement, the TA have greatly increase my interest in politics, increase my awareness of politics.''}

\end{itemize}

\href{mailto:hector.bahamonde@rutgers.edu}{Send me an email} to receive the latest teaching evaluations.

\vspace{1.5cm}
}{\bf Teaching References}

Professor Ross Baker\\\
Distinguished Professor\\\
Rutgers University\\\
\href{mailto:rosbaker@rci.rutgers.edu}{RossBaker@rci.Rutgers.edu}\vspace{20cm}



%\closing{{\color{white}empty here}}


%{\color{white}\encl{}}

%----------------------------------------------------------------------------------------

\end{letter}

\end{document}
