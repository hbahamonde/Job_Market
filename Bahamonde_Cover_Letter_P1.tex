% introduction
\vspace{-0.3cm}I am writing to apply for the position available at your institution. Currently, I am an assistant professor at O$'$Higgins University in Chile (a public research university). Since my wife is German, my main goal is to {\bf begin an academic career in Europe}. I am looking forward to start from the bottom of the academic hierarchy. 

% bio and my research in general
After receiving my PhD in Political Science from {\bf Rutgers University}-New Brunswick, NJ, U.S. in May 2017---where I studied under the direction of Robert Kaufman and Daniel Kelemen---I spent one year as a post-doctoral fellow at {\bf Tulane University}-New Orleans, LA, U.S. My research explores the economic and political origins of state capacity as well as the political economy of institutional development and the role of inequality on democratic development (clientelism). My methods include historical analyses, quantitative and experimental methods. I pay particularly attention to new ways of using digital information to construct natural experiments within the potential outcomes framework. 

In this cover letter I will highlight some details about my research and teaching agendas. 



% papers
{\bf Inequality and clientelism.} I kindly invite the reader to check both ``\href{https://journals.sub.uni-hamburg.de/giga/jpla/article/view/1121/1128}{Aiming Right at You: Group versus Individual Clientelistic Targeting in Brazil}'' (Journal of Politics in Latin America) and ``\href{https://doi.org/10.1057/s41269-020-00174-4}{Still for Sale: The Micro-Dynamics of Vote Selling in the United States, Evidence From a List Experiment}'' (Acta Politica). In both pieces I challenge the traditional role attributed to inequality when predicting clientelism. Both papers used top-notch statistical and experimental methods---and when possible, a novel dataset representative at the country level (collected thanks to a generous grant I received). Within the same research line, I have a number of papers in the pipeline. First, we are designing an {\bf economic experiment about vote selling and vote buying}. By implementing an ``auction game'' in the lab, the experiment recreates market conditions that exist between vote buyers and vote sellers. Finally, we are working on a \href{https://github.com/hbahamonde/Conjoint_US/raw/master/Bahamonde_Quininao_Conjoint.pdf}{paper} where we {\bf introduce machine learning methods to analyzing conjoint experimental data} in the context of clientelism (\emph{\input{/Users/hectorbahamonde/research/Conjoint_US/status.txt}\unskip}).


{\bf Economic inequality and democracy.} We are currently working on a paper that explores the relationship between state capacity and inequality. Using time-series econometrics we find in ``\href{https://github.com/hbahamonde/Inequality_State_Capacity/raw/main/Bahamonde_Trasberg.pdf}{\input{/Users/hectorbahamonde/research/Inequality_State_Capacity/title.txt}\unskip}'' ({\bf \emph{\input{/Users/hectorbahamonde/research/Inequality_State_Capacity/status.txt}\unskip}}) that democratic rule combined with high state infrastructural power produce \emph{higher} levels of income inequality over time. This relationship operates through the positive effect of high-capacity democratic context on investor confidence, FDIs, and financial development. In a second paper, and using a novel aerodrome usage dataset (publicly available digital information) we are constructing a natural experiment which looks at how Chilean elites were able to flight to their vacation houses, skipping lockdown policies. The \href{https://raw.githubusercontent.com/hbahamonde/Tobalaba/master/abstract.txt}{paper} shows how local authorities faced high political pressures to systematically ignore overseeing traffic control policies during the pandemic. We use regression discontinuity designs and in-depth interviews.

{\bf State building.} Using {\bf a novel dataset on historical earthquakes} to proxy state capacity, I have a \href{https://github.com/hbahamonde/Earthquake_Paper/raw/master/Bahamonde_Earthquake_Paper.pdf}{\texttt{working paper}} (\emph{\input{/Users/hectorbahamonde/Dissertation/Papers/Earthquake_Paper/status.txt}\unskip}) explains that sectoral conflicts between the landed and industrial elites fostered inter-elite compromises that lead to higher levels of state capacity over time. The paper seeks to analyze how economic structural transformations in Latin America helped states to make institutional investments that lead to the formation of states with higher capacity. I leverage fine-grained historical case study comparisons, sectoral economic outputs from 1900 to the present, Bayesian time-series econometric techniques, hazard models, and a novel earthquake dataset that covers sub-national death tolls from 1900 to the present to measure state capacity. This project builds on the fiscal sociology literature and the dual-sector economy model. 


% agenda: structure / behavior
As a comparativist and political economist, I believe that advanced methods should be used to answer big questions. In this sense, my research also has a disciplinary agenda. My scholarly work, for example tries to examine classic problems in comparative political development (e.g. state capacity, clientelism, etc.), while at the same time incorporating cutting-edge econometric and experimental techniques. Similarly, my working papers and experiments, are concerned with fundamental questions regarding democratic theory.

{\bf Digitalization: a research agenda.} Cities are constantly producing digital data. From MetroCard transactions, social media posts to Google searches. I feel particularly intrigued by new ways of exploiting such data so we can answer questions about inequality and clientelism. During my three years at WZB (IPI unit), I intend to move forward this agenda, particularly by publishing several papers that I have in the pipeline. First, I have two papers using conjoint data on clientelism (please check my \href{http://github.com/hbahamonde/Job_Market/raw/master/Bahamonde_CV.pdf}{\texttt{CV}}). All are at different stages of the publication process. Those digital data were collected via ICTs in an experimental way. Second, I have collected a digitalized population dataset on daily public transportation and COVID contagion in Santiago de Chile, the capital city of one of the most unequal countries in the world. I intend to study how low-income municipalities systematically bore the cost of the COVID pandemic. This is a really interesting project as it shows that (1) contagion thresholds to restrict mobility in low-income municipalities were higher and that (2) lockdown policies during daytime were systematically ignored by local authorities but heavily enforced during nighttime---which is curfew time, often used to protest against the government. These findings are relevant as they go in line with the recent ``politics of weakness'' literature. I use hazard models, synthetic control and regression discontinuity design methods. To conclude this portion of the statement, my research agenda on digitalization not only involves publishing these papers, but also organizing talks and presenting at conferences, as well as collaborating with other WZB affiliates.