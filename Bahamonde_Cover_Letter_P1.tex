% introduction
\vspace{-0.3cm}I am writing to apply for the position available in your institution. I am a post-doctoral fellow at the Center for Inter-American Policy \& Research (CIPR) at Tulane University. I received my Ph.D. in political science from Rutgers University - New Brunswick in May 2017, where I studied under the direction of Robert Kaufman.

% intro: my research in general
My {\bf research} explores the economic and political origins of state capacities as well as the political economy of institutional development and the role of inequality on democratic development. My methods include historical analyses, quantitative methods and experimental designs. While my research focuses predominantly on Latin America, my current and future research projects will be expanded to other developing and developed countries. A more detailed description of my scholarly work can be found in my \href{http://github.com/hbahamonde/Job_Market/raw/master/Bahamonde_Research_Statement.pdf}{\texttt{research statement}}.


% explain job market paper
Using {\bf a novel dataset on historical earthquakes} to proxy state capacities, my {\bf job market} \href{https://github.com/hbahamonde/Earthquake_Paper/raw/master/Bahamonde_Earthquake_Paper.pdf}{\texttt{paper}} (\emph{\input{/Users/hectorbahamonde/RU/Dissertation/Papers/Earthquake_Paper/status.txt}\unskip}) explains that sectoral conflicts between the landed and industrial elites fostered inter-elite compromises that lead to higher levels of state capacities overtime. 

% book // paper 1 and 2.
The paper is embedded into a {\bf larger research agenda and book manuscript} where I analyze how economic structural transformations in Latin America helped states to make institutional investments that lead to the formation of states with higher capacities. I leverage fine-grained historical case study comparisons, sectoral outputs from 1900 to the present, time-series econometric techniques, hazard models, and a novel earthquake dataset that covers sub-national death tolls from 1900 to the present to measure state capacities. This project builds on the fiscal sociology literature and the dual-sector economy model. In a \href{http://github.com/hbahamonde/IncomeTaxAdoption/raw/master/Bahamonde_IncomeTaxAdoption.pdf}{\texttt{first extension}} (\emph{\input{/Users/hectorbahamonde/RU/Dissertation/Papers/IncomeTaxAdoption/status.txt}\unskip}), I explain how the expansion of the industrial sector in post-colonial Latin America \emph{accelerated} the implementation of the income tax law. Leveraging the fiscal sociology paradigm, I explain how this institution was a state-\emph{making} institution. In a \href{https://github.com/hbahamonde/Negative_Link_Paper/raw/master/Bahamonde_NegativeLink.pdf}{\texttt{second extension }} (\emph{\input{/Users/hectorbahamonde/RU/Dissertation/Papers/NegativeLink/status.txt}\unskip}), I explain that the emergence of the industrial sector not only altered the balance of political power, but also fostered long-term economic growth. Building on the dual sector model of economic growth, I explain how `balanced economic growth' prevented any of the industrial and landowning classes to be the dominant elite, fostering inter-sectoral political compromises.

% papers
In addition to the book manuscript, I am currently expanding the findings of a {\bf series of papers} related to vote-buying and vote-selling, using both {\bf observational} and an original {\bf experimental} designs. 

{\bf Vote-Selling.} With the support of a generous grant, I designed two experiments in the U.S. out of a series of experiments to be fielded in Latin America for further comparison. I look at the tipping points where U.S. citizens prefer a monetary incentive rather than keeping their right to choose for whom to vote for. My identification strategy takes advantage of a \emph{list experiment} to capture non-biased answers on socially-condemnable/illegal behaviors (like vote-buying). In a separate study I designed a \emph{conjoint experiment} to identify which of Dahl's democratic dimensions should `fail' to predict individual propensities of vote-selling. Conjoint experiments allow researchers to directly isolate complex multi-dimensional concepts (such as \emph{support for democracy}) and observe which dimension(s) is/are associated to the outcome of interest (vote-selling). Preliminary results show that when the \emph{liberal} component, particularly the \emph{right to associate} fails, individuals are more likely to sell. In this sense, this project is very innovative as it departs from the common strategy of correlating the standard support for democracy question with vote-buying. My identification strategy is more comprehensive as it \emph{decomposes} ``support for democracy'' in several dimensions which are theoretically and substantively relevant (i.e., Dahl's democracy dimensions). 

{\bf Vote-Buying.} In addition to that I have a working \href{https://github.com/hbahamonde/Clientelism_paper/raw/master/Bahamonde_Clientelism_Paper.pdf}{\texttt{paper}} (\emph{\input{/Users/hectorbahamonde/RU/research/Clientelism_paper/status.txt}\unskip}) on vote-buying in Brazil. The paper starts by recognizing that there is not an agreement on whether parties target groups or individuals. In fact, most of the times, scholars assume that group-targeting and individual-targeting are interchangeable. What seems to be a major problem, however, is that scholars seem to base their decision on their research designs; ethnographers typically study how parties target \emph{individuals} while experimentalist scholars typically look at how parties target districts/municipalities/states (i.e. \emph{groups}). I developed and tested a theory where parties make use of simultaneous segmented targeting techniques. Groups are preferable by brokers when party machines need to secure higher levels of electoral support, relying on the economies of scale and spillover effects that these groups provide. However, individuals are better targets when they are more identifiable, that is when poor individuals are nested in non-poor contexts or when non-poor individuals are nested in poor groups. Interestingly, I find that non-poor individuals are also targeted. The paper uses observational data and matching methods and a short case study (Brazil) to confirm the assumptions made in the models.

% agenda: structure / behavior
As a comparativist and political economist, I believe that advanced methods should be used to keep answering important/big questions. In this sense, my research also has a disciplinary agenda. My scholarly work for example tries to examine classic problems in comparative political development (like state capacities) while at the same time incorporating cutting-edge econometric techniques. Similarly, my working papers and my experiments, which are framed within the behaviorist tradition, are concerned with fundamental questions regarding democratic theory. Incorporating the U.S. as a \emph{developed} case will allow me to systematically compare it with \emph{developing} Latin American countries.