%----------------------------------------------------------------------------------------
%	PACKAGES AND OTHER DOCUMENT CONFIGURATIONS
%----------------------------------------------------------------------------------------

\documentclass[11pt]{letter} % Default font size of the document, change to 10pt to fit more text

%\usepackage{newcent} % Default font is the New Century Schoolbook PostScript font 
%\usepackage{helvet} % Uncomment this (while commenting the above line) to use the Helvetica font

% packages
\usepackage{hyperref} % for links
\usepackage[bottom]{footmisc} % footnotes at the bottom.
\usepackage{color} % for colors

% Margins
\topmargin=-1in % Moves the top of the document 1 inch above the default
\textheight=9in % Total height of the text on the page before text goes on to the next page, this can be increased in a longer letter
\oddsidemargin=-10pt % Position of the left margin, can be negative or positive if you want more or less room
\textwidth=6.5in % Total width of the text, increase this if the left margin was decreased and vice-versa

%\let\raggedleft\raggedright % Pushes the date (at the top) to the left, comment this line to have the date on the right
\date{}


\usepackage{color} % for colors
\usepackage{hyperref} % for links

\hypersetup{
  colorlinks = true,
  urlcolor = blue,
  pdfpagemode = UseNone
}



\begin{document}

%----------------------------------------------------------------------------------------
%	ADDRESSEE SECTION
%----------------------------------------------------------------------------------------

\begin{letter}{} 

%----------------------------------------------------------------------------------------
%	YOUR NAME & ADDRESS SECTION
%----------------------------------------------------------------------------------------

\begin{center}
\large\bf Hector Bahamonde \\ % Your name
Rutgers University\\
\vspace{20pt} \hrule height 1pt % If you would like a horizontal line separating the name from the address, uncomment the line to the left of this text
89 George St. \\ New Brunswick, NJ 08901 \\ (732) 318-9650 \\ 
{\normalfont\normalsize\href{mailto:hector.bahamonde@rutgers.edu}{Hector.Bahamonde@Rutgers.edu}} \\
{\normalfont\normalsize\href{http://www.hectorbahamonde.com}{www.HectorBahamonde.com}}\\
{\normalfont \scriptsize{
% Date
\vspace{5mm}\today\\
Download last version \href{http://github.com/hbahamonde/Job_Market/raw/master/Bahamonde_Diversity_Statement.pdf}{\texttt{{\color{red}here}}}}} % Link to last version
\\
{\huge\vspace{6mm} Diversity Statement}
\end{center} 

%\signature{\vspace{6cm}h.b., fall 2016} % Your name for the signature at the bottom

%----------------------------------------------------------------------------------------
%	LETTER CONTENT SECTION
%----------------------------------------------------------------------------------------
\opening{} 
\vspace{-1.5cm}
Rutgers is one of the most diverse universities in the United States. Being a teaching assistant of large classes has given me the opportunity to experience diversity. Diverse environments not only make individuals more tolerant, flexible and educated. They also force us, in a good way, to develop teaching philosophies that are flexible, considerate and interesting for diverse student bodies. 

There are many types of diversity, such as economic, cultural, sexual and political. We, as social scientists, should know that all these types are part of our daily life as researchers and instructors. Everything we do, will most likely be related to economic inequality, racial politics or gender representation. Everything we say is subject to debate and tension. These disputes should be treated even more carefully, specially if our audience is diverse. It is not about not making political jokes. It is more about answering the following questions: \emph{How can I make today's lecture interesting enough, so my students engage not only with me but with all their classmates? How should I present this contentious issue related to race, for example, in a way that at the end of class, my students know that there is still much more to be done?} In my experience, the answers have less to do with teaching ``the facts.'' They have to do with knowing how to expose students to complex issues. \emph{Should the rich be taxed to help the poor? Was the atomic bomb a reasonable measure? Are all democratic outcomes ``good''?} (Hitler was democratically elected) \emph{Should rich states help poor states?} (the economic literature says ``no''), etc. All these topics can be answered from so many angles. Moreover, \emph{students will answer based on their own economic, cultural, sexual and political backgrounds}. At the end of the day, I know that I did my job well if I know that my students felt \emph{curious} about these issues, even at the cost of leaving the classroom in an atmosphere of ``uncertainty.''

All and all, as instructors, I think diversity plays always in our favor. By all means, it makes lecturing more interesting. But it also plays in favor of our  students. Diversity exposes them to different perspectives. However, even more importantly, diversity forms better citizens better able to engage in a diverse world. That is why I think we should always ``take advantage'' of diversity.



\closing{{\color{white}empty here}}



%----------------------------------------------------------------------------------------

\end{letter}

\end{document}
